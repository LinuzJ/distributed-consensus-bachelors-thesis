\documentclass[english, 12pt, a4paper, sci, utf8, a-1b, online]{aaltothesis}



\usepackage[parfill]{parskip}
\usepackage{float}
\usepackage{graphicx}
%% Math fonts, symbols, and formatting
\usepackage{amsfonts,amssymb,amsbsy,amsmath, enumitem}
\usepackage{xcolor}
\usepackage{amsthm}
\usepackage{svg}

\newtheorem{theorem}{Theorem}[section]
\usepackage{theoremref} 
\newtheorem{corollary}[theorem]{Corollary}
\newtheorem{lemma}[theorem]{Lemma}
\newtheorem{proposition}[theorem]{Proposition}
\newtheorem{result}[theorem]{Result}
%Definition environments:
\theoremstyle{definition}
\newtheorem{definition}[theorem]{Definition}
%Remark environments:
\theoremstyle{remark}
\newtheorem{remark}[theorem]{Remark}%\declaretheorem{Definition}
%\declaretheorem{Lemma}
%\declaretheorem{Theorem}

%% THESIS INFO
\degreeprogram{Computer Science}

\major{Computer Science}

\code{SCI3028.kand}

\univdegree{BSc}

\thesisauthor{Linus Jern}

\thesistitle{Distributed consensus by population protocols}

\place{Espoo}

\date{\today}

\supervisor{Postdoctoral researcher  \ Francesco d'Amore}

\advisor{Postdoctoral researcher  \ Francesco d'Amore}


%% Aaltologo: syntax:
%% \uselogo{aaltoRed|aaltoBlue|aaltoYellow|aaltoGray|aaltoGrayScale}{?|!|''}
\uselogo{aaltoBlue}{''}


%% THE ENGLISH ABSTRACT:
%% Thesis keywords:
\keywords{Population protocol\spc Consensus Problem\spc Distributed algorithms\spc Majority consensus\spc Byzantine agreement}

%% The abstract text. This text is included in the metadata of the pdf file as well
%% as the abstract page.


\copyrighttext{Copyright \noexpand\copyright\ \number\year\ \ThesisAuthor}
{Copyright \copyright{} \number\year{} \ThesisAuthor}

%------------------------------------------------------------------------------------------------

\newcommand{\linus}[1]{{\color{red} \textbf{Linus}: #1}}

\newcommand{\myabs}[1]{{\lvert #1 \rvert}}

\begin{document}

%% Create the coverpage
\makecoverpage


%% Typeset the copyright text.
\makecopyrightpage

\supervisor{University Lecturer  \ Henrik Wallén}


\pagenumbering{gobble}

%% ENGLISH ABSTRACT
\begin{abstractpage}[english]
    The field of population protocols has seen recent progress in solving the consensus problem. In this literature review some of the significant population protocols solving these issues, as well as recent improvements to these are discussed. The Undecided State Dynamics Protocol solves the approximate majority problem in $O(n \log n)$ time with high probability provided that the initial majority is at least $\omega(\sqrt{n} \log n)$. It is however limited to $k = 2$, which is two possible opinions. Recent progress improved this protocol to solve the approximate majority problem with $k > 2$ initial opinions. Additionally, the Nonuniform-Majority protocol that solves the exact majority problem for $k = 2$ with high probability is discussed. This protocol is capable of reaching consensus in $O(\log n)$ time, however, the memory usage is not constant using $O(\log n)$ states ($\log \log n + O(1)$ bits of memory). The generalization of this protocol to solve the exact majority problem for $k > 2$ is also discussed.
\end{abstractpage}

%% Force new page so that the Swedish abstract starts from a new page
\newpage

%% SWEDISH ABSTRACT.
\thesistitle{Distribuerad konsensus med hjälp av populations protokoller}
\supervisor{Universitets Lektor  \ Henrik Wallén}
\advisor{Postdoktoral forskare  \ Francesco d'Amore} 
\degreeprogram{Datateknik}
\major{Datateknik}
\keywords{Populationsprotokoll\spc Konsensusproblem\spc Distribuerade algoritmer\spc Majoritetsövernskommelse\spc Byzantine övernskommelse}
\begin{abstractpage}[swedish]
Det har forskats länge om distribuerad konsensus och det fundamentala problemet fortfarande lika relevant som när forskningen började på 80-talet. De fundamentala teorierna bakom det distribuerade konsensusproblemet, eller konsensusproblemet, är relevanta för viktig digital infrastruktur som vårt moderna samhälle förlitar sig på. En stor del av datan som är lagrad på internet idag är lagrad i distribuerade databaser vars varje transaktioner måste uppnå konsensus för lyckat uföras. 

Den nyaste trenden inom forskningen om konensusproblemet är populationsprotokoll. Inom forskningsområdet för populationsprotokoll har det nyligen gjorts framsteg i att lösa konsensusproblemet. I denna litteraturstudie diskuteras några av de viktigaste populationsprotokollen som löser olika varianter av konsensusproblemet, samt nyligen gjorda förbättringar till dessa protokoll. Undecided State Dynamics Protocol är en av de första anvädningarna av populationsprotokoll för att lösa det ungefärliga majoritetsproblemet. Detta protokoll löser problemet i fråga på $O(n \log n)$-tid med hög sannolikhet förutsatt att den ursprungliga majoriteten är minst $\omega(\sqrt{n} \log n)$. Detta protokoll är dock begränsat till endast två möjliga åsikter för noderna i systemet. Detta skrivs formellt som $k = 2$, där $k$ är antalet åsikter. Nyligen har dock framsteg gjorts för att förbättra detta protokoll så att det löser det ungefärliga majoritetsproblemet med $k > 2$ ursprungliga åsikter. Denna generalisering av protokollet löser problemet inom en tid på $O(kn \log n)$. Utöver det ungefärliga majoritetsproblemet undersöks även det exakta majoritetsproblemet samt protokoll som löser detta problem. Det optimerade protokollet Nonuniform-Majority löser det exakta majoritetsproblemet för $k = 2$ med hög sannolikhet. Detta protokoll kan nå konsensus inom en tid på $O(\log n)$, men minnesanvändningen är inte konstant eftersom det krävs att en nod kan lagra upp till $O(\log n)$ tillstånd ($\log \log n + O(1)$ bits minne). Några månader efter att Nonuniform-Majority protokollet publicerades förbättrades det till en generalisering som är kapabel att lösa det exakta majoritetsproblemet för $k > 2$ inom $O(\frac{n}{x_{max} \cdot \log n + \log^2 n})$ interaktioner ($O(k \cdot \log \log n + \log n)$ minne). 

\end{abstractpage}

%% TABLE OF CONTENTS
\thesistableofcontents

\cleardoublepage

\pagenumbering{arabic}
%% INTRODUCTION
\section{Introduction}
\thispagestyle{empty}
%% Intro
A distributed system can be described as a collection of computing entities, also called agents, connected in a shared network which perform computations and exchange messages with each other in order to reach some global task \cite{Coulouris_systems_2005}. Distributed systems are a crucial part of the modern digital infrastructure that the world heavily relies on. They enable large, scalable and low-latency systems for a wide range of applications and audiences. One such system is a distributed database where the responsibility of storing the data is shared across a set of computers. To achieve a shared goal among the agents, they often need to reach a common agreement. An agreement here means that all agents involved have reached a decision on whether to do something or not. Distributed consensus is a fundamental problem in distributed computing that asks the distributed system to achieve an agreement respecting some properties. Consensus is crucial for reaching and maintaining data consistency, ensuring fault tolerance and enabling cooperation between the agents in the system. \cite{Lynch_distributed_2017}

%% Byzantine generals problem
A common way of formalizing the problem of distributed consensus is through the Byzantine Generals Problem, introduced by Leslie Lamport in \cite{lamportByzantineGeneralsProblem}. In this definition there is a number of generals trying to coordinate an attack on a city. Some of the generals may prefer to attack and some may prefer to retreat, and they have to reach a shared agreement on whether to attack or not. Additionally, there may be further complications like treasonous generals spreading sub optimal information and the fact that the only way for the generals to talk to each other is to send a messenger on foot, who might get captured or injured while delivering message. 
%% Byzantine fault
The previously mentioned condition that describes the possibility of a part of the system failing (messenger getting injured or mischievous general in Byzantine terms) is referred to as a Byzantine fault. In short, a Byzantine fault is a fault that causes a different results to arrive to different observers. Or when a sending agent sends out some other message than it should. \cite{driscollRealByzantineGenerals2004}. 
This formalization provides an easily understandable base with which one can visualize the different solutions.

%% Syncrounous vs Asynchronous problem
The way in which agents communicate within a distributed system is another important aspect that should be taken into account when handling the distributed consensus problem. The problem may be modelled in a synchronous or an asynchronous system. A synchronous system is a system where agents use a global clock (or they have perfectly synchronized local clocks), while in an asynchronous system, each agent has its own local clock. The agents in the system perform computational tasks when the clock they are referring to ticks one time unit forward. Asynchronous systems introduce additional complexity and cannot solve the distributed consensus problem in general \cite{fischerImpossibilityDistributedConsensus}.

%% Existing solutions
To solve the consensus problem, there are multiple established algorithms, both in the synchronous and asynchronous systems. Examples of established algorithms are Paxos \cite{lamportFastPaxos2006}, Raft \cite{ongaroSearchUnderstandableConsensus} and Practical Synchronous Byzantine Consensus \cite{renPracticalSynchronousByzantine} in the synchronous setting and pBFT \cite{castroPracticalByzantineFault} and population protocols \cite{aspnesIntroductionPopulationProtocols2009} in the asynchronous setting.

%% Brief paxos
The paxos algorithm mentioned above is one of the first algorithms introduced to solve the consensus problem, even with the presence of failures. It was introduced by Leslie Lamport  in \cite{lamportPartTimeParliment1998}. At a high level, the Paxos algorithm is divided into two phases. In the first phase, a proposer agent sends out its initial value to the other agents, who respond with an acceptation or an rejection. In the second phase, the proposer agent sends out another proposal, including the responses from the first phase, and repeats this process until the majority if the agents in the system have accepted a common value. 

%% Brief Population protocol
Population protocols are more modern consensus algorithms in the asynchronous setting. Population protocols are theoretical models used for modelling collections of moving agents that are capable of interacting and performing computation. Their purpose are for the collection of agents to converge towards a correct output value. The initial input values for the agents are distributed to the agents in the system after which adjacent pairs of agents can exchange information. The agents move around in an unpredictable manner, however subject to some fairness constraints and computation, the collection of agents will converge to the correct output state. \cite{aspnesIntroductionPopulationProtocols2009} 

%% Bried reasoning for focusing on PPs
Population protocols are a modern model in distributed computing that were introduced by Dana Angluin in \cite{angluinComputationNetworksPassively2006}. The application of population protocols vary from sensor networks to the interactions of molecules in theoretical chemistry. \cite{aspnesIntroductionPopulationProtocols2009} The wide application possibilities and the amount of recent breakthroughs made in the field (e.g. \cite{dotyTimeSpaceOptimal2022}, \cite{bankhamerPopulationProtocolsExact2022}) makes population protocols highly interesting and is the reason this thesis will, in addition to giving a overview of the consensus problem, focus on population protocols.

The goal of this thesis is to clearly define what the problem of distributed consensus is, as well as discuss the existing approaches and algorithms to the problem. Afterwards, we will dig into the modern asynchronous population protocol algorithms that have widely investigated in recent years. 

This thesis is divided into two parts. The first part will discuss the introduction of population protocols, their place in the domain of consensus algorithms, as well as go through the first population protocol presented by Angluin \cite{angluinSimplePopulationProtocol2008}, as well as the generalized version of that same protocol \cite{AspnesFastConverganceOfKOpinion2023}. The second part will cover optimizations of the general population protocols. Lots of research about optimizing different dimensions of population protocols has been done in recent years and one optimization protocol will be discussed in this section. 

There are many literature surveys on specific subdomains of the distributed consensus problem, however not many providing a general overview of the complete topic. As the literature on this topic is large, we will cover some important parts of it with a strong focus on population protocols.


%% SYMBOLS AND ABBREVATIONS
%% SYMBOLS AND ABBREVIATIONS
\section{Preliminaries}

%%------------------------------------------------------------------------------------
%% BIG O
\subsection{Big O notation}

 The Big $O$ is a notation that describes an upper bound to the asymptotic behaviour of functions, when the argument goes to infinity. In computer science the Big $O$ notation is used to analyze the time and space complexity of algorithms. The Big $O$ notation is a function of \inlineMath{n}, where \inlineMath{n} is the number of items handled in the algorithm. Described informally using the equation \inlineMath{f(n) = O(g(n))}, \inlineMath{f(n)} is positive and smaller than some constant multiplied with \inlineMath{g(n)}.
 
 \begin{definition}
 We write \inlineMath{f(n) = O(g(n))} if there exist two constants
 \inlineMath{c > 0} and \inlineMath{k > 0} such that \inlineMath{0 \leq f(n) \leq cg(n) \ \forall n \geq k}.
 \end{definition}

In addition to Big O, used for upper bounds, there is the Big Omega ($\Omega$) and little omega ($\omega$) notation. These are used to describe the asymptotic lower bounds of functions. 

 \begin{definition}
 We write $f(n) = \Omega(g(n))$ if there exists a constant
 \inlineMath{c > 0} and \inlineMath{k > 0} such that $f(n) \geq cg(n) \geq 0 \: \forall \: n \geq k$
 \end{definition}

 \begin{definition}
We write $f(n) = \omega(g(n))$, if there exists a constant
 \inlineMath{k > 0}  such that $g(n) \neq 0$ for all $n > k$ and $ \lim_{x \to \infty} f(n) / g(n) = + \infty$
 \end{definition}

Big Theta ($\Theta$) is used to define a asymptotic tight bound on some function. 

 \begin{definition}
    We write $f(n) = \Theta(g(n))$, if there exists three positive constants, $c_1, c_2, n_0$ such that $c_1 g(n) \leq f(n) \leq c_2 g(n) \ \forall n \geq n_0$
 \end{definition}



%%------------------------------------------------------------------------------------
%% Distributed system
\subsection{Distributed system} \label{preliminaries_distributed_system}

A distributed system is a set of networked computers, which coordinate their actions through communicating by messages. We can define this system as a set of nodes, connected in a network, that collectively coordinate and execute tasks.

Let the communication network of the distributed system be modeled as a graph \inlineMath{G = (V, E)}, where \inlineMath{V} is the set of vertices (or nodes), meaning the computing entities, or processes, of the system and \inlineMath{E} is the set of edges in the system that make up the communication links between the edges. The terms agent, node will be used interchangeably. 

Each individual node, or process, in the set of nodes \inlineMath{V} has a state from a set $\Sigma$ of admissible states. Let the state of the system be  \inlineMath{S = (S_1, S_2, ...., S_N)}, where \inlineMath{N} is the amount of nodes in the system. The behaviour of the system can be a set of rule of how the nodes interact with each other, altering the individual internal states of nodes and edges. These rules can be formalized as a set of functions \inlineMath{F}, that map the current state of the system \inlineMath{S} to a new state \inlineMath{S'}.

Define a distributed system as a tuple \inlineMath{(V, E, S, F)}, where \inlineMath{V} is the set of vertices (or nodes), \inlineMath{E} is the set of communication links between nodes, \inlineMath{S} is the current state of the system and \inlineMath{F} is the set of functions that define the behaviour of the system.

%%------------------------------------------------------------------------------------
%% CONSENSUS PROBLEM
\subsection{Consensus problem}

The consensus problem asks to design a protocol that requires all computing entities, called agents, in a system, to agree on a binary value. This system of agent may include faulty agents, that may fail or produce faulty messages. The challenge is to make all non-faulty agents to have a shared understanding of the binary value in question, even with the presence of faulty agents. 

In order to reach consensus, each node in the set \inlineMath{V}, which is the set of nodes in the system, begins by \emph{proposing} its state. Let the value proposed be \inlineMath{v_i}. The nodes then communicate and share their initial proposals. The nodes then decide on a decision value \inlineMath{d_i} and change their states to it. The nodes are now in the \emph{decided state}, from which they can no longer return nor change the value of \inlineMath{y_p}. The requirements of a consensus algorithm are that, for each execution of it, these conditions should hold:

\begin{itemize}[label={}]
  \item \emph{Termination:} All correct nodes eventually sets the value of their output register and reach a decided state.
  \item \emph{Agreement:} All correct nodes share the same state: if \inlineMath{V_i} and \inlineMath{V_j} are correct nodes and are in the \emph{decided state}, their corresponding states \inlineMath{S_i} and \inlineMath{S_j} share the same state \inlineMath{y_i = y_j}.
  \item \emph{Integrity:} If all correct nodes proposed the same value, then any correct node that is in the \emph{decided state} has chosen that value.
\end{itemize}


\begin{figure}[H]
    \centering
    \includesvg[width = 0.7\textwidth]{figures/consensus_problem.svg}
    \caption{Consensus for three processes}
    \label{fig:ConsensusProblem}
\end{figure}


See figure \ref{fig:ConsensusProblem} for a highly simplified version of a consensus algorithm. Node $v_1$ and $v_2$ propose \emph{proceed} to the algorithm and $v_3$ proposes \emph{abort} but immediately after crashes. The two nodes remaining nodes are correct and decide to \emph{proceed}.

%%------------------------------------------------------------------------------------
%% MAJORITY CONSENSUS
\subsection{Majority Consensus}

In the classical consensus problem, consensus is reached when all correct nodes eventually reaches a decided, or finished state. This value may be whatever, as long as all of the correct nodes share the same value in their decided state. However, \emph{majority} consensus requires the algorithm to agree on the initially most frequent value.

\begin{definition}
In a system with \inlineMath{n} agents and \inlineMath{k} states. Majority consensus is reached when the agents have agreed on the most frequently occurring initial state. 
\end{definition}
 
 \emph{Exact} majority consensus is a requirement that the agreement must me made correctly even though the initially most frequent and second most frequent only differ in size by 1. In literature, the terms \emph{majority} and \emph{plurality} are often used interchangeably. \emph{Majority} will be used in this thesis for consistency.

\emph{Approximate} majority consensus is the requirement of the initial majority state having at least a certain amount 


%%------------------------------------------------------------------------------------
%% ASYNC AND SYNC SYSTEMS
\subsection{Asynchronous and synchronous systems}

A distributed system can be modeled in a synchronous or an asynchronous setting. In a synchronous system, all agents essentially use the same clock. The algorithms used in a synchronous system assume that steps take place in discrete rounds. A agent can be assumed to be faulty if no message has been received from it at the end of the round. In a asynchronous system on the other hand, agents can send messages at arbitrary times, meaning that if an agent fails, it is indistinguishable from an agent responding slowly.


%%------------------------------------------------------------------------------------
%% POPULATION PROTOCOLS
\subsection{Population protocol}
 A population protocol is a theoretical model used for modelling collections of agents capable of moving, interacting and computation. The goal of the protocol is for the collection of agents to converge towards a correct output value. In the basic population protocol model, an input value is distributed to the the collection of agents. Agents have pairwise interaction in the order set by a scheduler, subject to some fairness guarantee. Each agent in the collection is a type of finite state machine and the protocol for the system describes how the interaction between two agents change their respective states. No failures occur for the agents in the system. The output values of the agents change over time and eventually, they must converge to the correct output value for the set of input values initially distributed to the agents \cite{aspnesIntroductionPopulationProtocols2009}. 

 A protocol is formally defined by
 \begin{itemize}
     \item \inlineMath{Q}, a finite set of possible states for an agent,
     \item $\Sigma$, a finite input alphabet,
     \item $\zeta$, an input map $\Sigma \mapsto Q$, where $\zeta(\sigma)$ represents the initial state of an agent and the input to that agent is $\sigma$,
     \item $\omega$, an output map $Q \mapsto Y$, where $Y$ is the output range and $\omega(q)$ represents the output value of an agent in state $q$,
     \item $\delta \subseteq Q \times Q \mapsto Q \times Q$, a transition relation that describes interactions between agents.
     
 \end{itemize}

A computation following a protocol defined as above, proceeds as follows. Let the system have \inlineMath{n} \emph{agents}, where \inlineMath{n \geq 2}. And let the computation take place in the system mentioned previously. The input value for each agent in the system is a value from $\Sigma$. The initial state is determined by using $\zeta$ on all agents input values. Let the \emph{configuration} of the system be a vector \inlineMath{C} that contains all the states of the agents. 

A protocol is made up of many executions. An execution alters the \emph{configuration} of the system through pairwise interaction between agents. During each execution step, a pair of agents \inlineMath{(v, w)} are chosen at random. Interactions are usually asymmetric, meaning one agent (\inlineMath{v}) acts as the \emph{initiator} and the other \inlineMath{(w)} acts as the \emph{responder}. The chosen pair pf agents have the states \inlineMath{q_1} and \inlineMath{q_2}. The agents with states \inlineMath{q_1} and \inlineMath{q_2} can change into the states \inlineMath{q_1'} and \inlineMath{q_2'} if the interaction \inlineMath{(q_1, q_2, q_1', q_2')} is in the transition relation $\delta$. This interaction could also be described using the notation $(q_1, q_2) \mapsto (q_1', q_2')$.  If \inlineMath{C} and \inlineMath{C'} are \emph{configurations} in the system, \inlineMath{C \rightarrow C'} means that \inlineMath{C'} can be reached from \inlineMath{C} through a single interaction. The previously mentioned \emph{execution} of the protocol is an infinite sequence of configurations \inlineMath{C_0, C_1, C_2, ...} where \inlineMath{C_0} is the initial configuration and $C_i \rightarrow C_{i+1} \forall i \geq 0$. 
\\\\
The pairwise interactions between agents occur in an unpredictable order. The sequence of interactions can be thought of as an adversary, under whose decisions the protocol must work correctly. However, in order for significant computations to take place, some restrictions must be places on the adversary scheduler of the system. Otherwise, the adversary scheduler could divide agents into two isolated groups and only schedule pairwise interactions with agents isolated in their own groups.  
\\\\
As stated previously, the systems scheduler is subject to some \emph{fairness} condition. This fairness condition ensures that the scheduler cannot avoid a certain step indefinitely. Expressed more formally: Let \inlineMath{C} be a configuration that exists an infinite amount of times in an execution. If \inlineMath{C \mapsto C'}, then the configuration \inlineMath{C'} also must exist an infinite amount of times in the execution. This can be summarised as the requirement: all potential configurations that can be reached, will eventually be reached.
\\\\ 
The correctness of a population protocol can also be compared to the execution of the algorithm. During the execution of a population protocol, if the state of any given agent at any given time is \inlineMath{q}, then its output value will be \inlineMath{w(q)}. This means that the output of an agent may change during an execution. According to the fairness constraint explained previously, the scheduler of the protocol may schedule arbitrary interactions only up to a certain point. Correctness can also be phrased in a similar way: all agents must eventually produce the correct output, and continue doing so after that point in time.

Performance of population protocols are measured in \emph{time complexity} and \emph{space complexity}. Time complexity in a population protocol is how many interactions are required to reach the final correct state. The standard way to express this is in \emph{parallel time}, which is the total amount of interactions done in the system divided by the population of the system $n$. Space complexity for a population protocol is measured in how many different values an agent in the system must be able to store. 



%% Early populaiton protocols
\section{Population Protocols}
%% TEXT %%
\subsection{Background}
\linus{TODO: Explain background to consensus problem and its relation to population protocols}

The consensus problem is fundamental challenge in the field of distributed computing. The problem has been studied extensively in literature and there are multiple popular algorithms 


\subsection{3-state approximate majority protocol} \label{3stateApproximatemajority}
%% ---------------------------------------------------
%% INTRODUCTION
\subsubsection{Introduction} \label{angluinIntroduction}

One of the first population protocol for approximate majority consensus was presented for a simple 3-state system by Angluin et al. in \cite{angluinSimplePopulationProtocol2008}. The protocol is shown to converge to a consensus in \inlineMath{O(n \log n)} interactions with high probability. It is also shown that the output value is correct, meaning it matches the initial majority, with high probability if the initial net majority is $\omega(\sqrt{n} \log n)$. \fda{Maybe define with high probability?}

The protocol is a simple 3-state protocol, meaning \inlineMath{Q = \{x, y, b\}}. The purpose of the protocol is to make all agents decide on the initial majority state, either \inlineMath{x} and \inlineMath{y}. The extra state \inlineMath{b} is a an \emph{blank} state. The idea of the protocol is that when two agents with different states meet in an interaction, the \emph{responder} abandons its state and enters the blank state \inlineMath{b}. If an agent with the blank state \inlineMath{b} interacts with another agent, it adopts the other agents state, assuming the other agent does not also have the blank state \inlineMath{b}. Because the collisions between agents are chosen by an adversary, the interactions between agents with the opposite state are equally balanced. However, because an agent with the blank state is more likely to interact with an agent with the initial majority state, the initial majority will increase until all agents have the same state. Once all agents have converged to the same state, the protocol is finished and the system has reached consensus. This protocol is also known as the \emph{Undecided State Dynamics} or \emph{USD} protocol.

Furthermore, Angluin shows that with high probability that the inclusion of \inlineMath{o(\sqrt{n})} Byzantine agents cannot notably delay the protocol converging to a state where the majority of non-Byzantine agents have the correct state. Byzantine agents are agents capable of appearing as any state, regardless of their previous interaction and initial state. The inclusion of these Byzantine agents can, however, keep a small part of of the non-Byzantine agents confused. Additionally, after exponential time on average, the Byzantine agents can eventually push the system to a stable incorrect state, where all of the non-Byzantine agents are blank. In the case that \inlineMath{z} Byzantine agents are included amongst the population \inlineMath{n} of normal agents, any execution select a pair at random from the combined population of \inlineMath{n} and \inlineMath{z}.

\subsubsection{Notations} \label{notations_angluin_usd}

The agents in the system can have the states in \inlineMath{Q = \{x, y, b\}}, where \inlineMath{b} is the \emph{blank} state. We can map out all possible interactions between agents of different states in \inlineMath{Q} in the following table

\begin{figure}[H]
    \centering
    \begin{tabular}{|c | c | c | c|} 
     \hline
      & \inlineMath{x} & \inlineMath{b} & \inlineMath{y} \\ [0.5ex] 
     \hline
     \inlineMath{x} & \inlineMath{(x, x)} & \inlineMath{(x, x)} & \inlineMath{(x, b)} \\ 
     \hline
     \inlineMath{b} & \inlineMath{(b, x)} & \inlineMath{(b, b)} & \inlineMath{(b, y)} \\
     \hline
     \inlineMath{y} & \inlineMath{(y, b)} & \inlineMath{(y, y)} & \inlineMath{(y, y)} \\
     \hline
    \end{tabular}
    \caption{Potential interactions for \inlineMath{Q}}
    \label{fig:QInteractions}
\end{figure}

All interactions modify the \emph{responders} state. This makes the protocol \emph{one-way}. Notable is also that not all interactions changes the states. For example \inlineMath{xx} does not change any state. The interactions that do change states are \inlineMath{xy, yx, yb, xb}.The system can be in three stable configurations \inlineMath{C}: all \inlineMath{b}'s, all \inlineMath{x}'s or all \inlineMath{y}'s. The first one of the previously named configurations cannot be reached from any configuration containing anything else than \inlineMath{b}'s. The two latter configurations are also stable and from every configuration made up of not only \inlineMath{b}'s, one of them can be reached.

Let \inlineMath{x_t, y_t} and $b_t$ be the number of \inlineMath{x, y} and \inlineMath{b} after \inlineMath{t} interactions. Let the \emph{convergence time} $\tau_*$ of the protocol be the first time \inlineMath{t} at which \inlineMath{x_t = n} or \inlineMath{y_t = n}.

To reduce size of explanations, we also define:
\begin{description}
    \centering
    \item[] \inlineMath{u = x - y}
    \item[] \inlineMath{v = x + y = n - b}
    \item[] \inlineMath{g = 1/n(n - 1)}
\end{description}

This definition of $u$ and $v$ exposes the symmetry between $x$ and $y$. The variable $g$ gives us the opportunity to calculate the probability of an interaction happening. For example $gvb$ gives the probability of a non-blank agent interacting with a blank agent. We define a configuration space with four regions: a central space where the $x, y$ and $b$ are quite evenly balanced and three corner regions where the amount of agents with status $x, y$ or $b$, depending on the corner, is almost $n$. The configuration space is visualized in Figure \ref{fig:configurationSpace}.

\begin{figure}[H]
    \centering
    \includesvg[width = 0.65\textwidth]{figures/configuration_space.svg}
    \caption{Configuration space}
    \label{fig:configurationSpace}
\end{figure}


To define interactions and sums of interactions, indicator variables are used. For example, let $I^{vb}_t$ be any interaction between a $xy$ or $yx$ pair at time $t$. All indicator variables $I_t$ have a corresponding sum variable $S_t = \sum^t_{i=1}I_i$ that describes the total amount of times this indicator event has occurred up until time $t$. The following indicator variables will be used:

\begin{table}[H]
    \centering
    \begin{tabular}{|c c c|}
     \hline
     Indicator & Sum & Event \\ 
     \hline
     \inlineMath{I^{vb}} & \inlineMath{S^{vb}} & $xb$ or $yb$ interaction \\
     \hline
     \inlineMath{I^{xy}} & \inlineMath{S^{xy}} & $xy$ or $yx$ interaction \\
     \hline
     \inlineMath{I^{b}} & \inlineMath{S^{b}} & Interaction in $b$ corner with $b \geq (7/8)n$ \\
     \hline
     \inlineMath{I^{x}} & \inlineMath{S^{x}} & Interaction in $x$ corner with $x \geq (7/8)n$  \\
     \hline
     \inlineMath{I^{y}} & \inlineMath{S^{y}} & Interaction in $y$ corner with $y \geq (7/8)n$  \\
     \hline
     \inlineMath{I^{c}} & \inlineMath{S^{c}} & Central interaction: $I^b = I^x = I^y = 0$ \\
     \hline
     \inlineMath{I^{z}} & \inlineMath{S^{z}} & Interaction with a Byzantine agent \\
     \hline
    \end{tabular}
    \caption{Interaction and sum notations}
    \label{fig:QInteractions}
\end{table}

%% ---------------------------------------------------
%% RESULTS FROM SIMPLE PROPROT
\subsubsection{Results}

The main results that Angluin et al. presents in \cite{angluinSimplePopulationProtocol2008} were briefly described in section \ref{angluinIntroduction}. The protocol itself is very simple, agents interacting according to the rules set in figure \ref{fig:QInteractions} will converge towards a consensus. By the proofs presented in \cite{angluinSimplePopulationProtocol2008}, a high-probability bound for the total interactions done before reaching convergence can be presented with the following theorem.

 \begin{theorem}\thlabel{theorem1}
    \textit{Let $\tau_*$ be the time at which \inlineMath{x = n} or \inlineMath{y = n} first holds. Then for any fixed \inlineMath{c > 0} and sufficiently large $n$,}
    \begin{align}
        \centerline{$\Pr[\tau_* \geq \log n + 6773cn \log + 2552n] \leq 5n^{-c}$.}  \label{angluinTheorem1} 
    \end{align}
 \end{theorem}

\ref{angluinTheorem1} in \thref{theorem1} gives us is a high probability upper bound for $\tau_*$. The constants in this theorem are large and it is shown in \cite{angluinSimplePopulationProtocol2008} that simulations produce much smaller constants, however, theorem \ref{theorem1} gives us a theoretical upper bound for the protocol in a system with only non-Byzantine agents. This convergence is quite quick, but does however require a initial margin between the population size of $x$ and $y$ in the initial state of the system. The next main result Angluin et al. shows in \cite{angluinSimplePopulationProtocol2008} is the lower bound for the difference between initial population sizes that the protocol tolerates, still producing a high probability convergence. This lower bound is presented in \thref{theorem2}


 \begin{theorem}\thlabel{theorem2}
    \textit{With high probability, the 3-state approximate majority protocol converges to the initial majority value if the difference between the initial majority and initial minority populations is } $\omega(\sqrt{n} \log n)$.
 \end{theorem} 

 Angluin also explores the correctness of the protocol in the case where the protocol has an epidemic-triggered start in \cite{angluinSimplePopulationProtocol2008}. The difference here compared to the systems previously discussed is that agents have an additional \emph{active/inactive} property. The initial state contains some subset of active and some subset of inactive agents. The inactive agents gets recruited into the computation of the protocol by active agents. It is shown in \cite{angluinSimplePopulationProtocol2008} that to guarantee a convergence to the correct value, a larger initial majority is needed, meaning a larger lower bound. The third main Angluin et al. presents in \cite{angluinSimplePopulationProtocol2008} is \thref{theorem3}.

 \begin{theorem}\thlabel{theorem3}
    \textit{Let $\epsilon > 0$. If the difference between the initial majority and initial minority populations is $\Omega(n^{3/4+\epsilon})$ and there is exactly one active agent, then with high probability, the epidemic-triggered approximate majority protocol converges to the initial value.}
 \end{theorem} 

 The inclusion of Byzantine agents in the system is also studied in \cite{angluinSimplePopulationProtocol2008}. A new group of agents $z$ is introduced into the system, where $z = o(\sqrt{n})$. Angluin shows that, even with the presence of $z$ Byzantine agents, the protocol will converge in $O(n \log n)$ interactions. This, however, requires some modifications to both the initial majority requirements as well as the definition of convergence. Due to the fact that any Byzantine agent at any time can shift the state to something that would not be possible in the presence of only non-Byzantine agents. A Byzantine agent could shift to a $y$ agent while there are no non-Byzantine $y$ agents left. This forces the convergence acceptance criteria to be slackened a bit, allowing some non-Byzantine agents to be in the incorrect state. Additionally, there exists a possibility where the inclusion of Byzantine agents leads the system to a stable state with only $b$. Angluin shows that even with these additional complexities, the probability of reaching the $b$ corner (as visualized in Figure \ref{fig:configurationSpace}) is small and that reaching the $x$ and $y$ corners still does not require that many interactions. The convergence time with the inclusion of Byzantine agents is presented by Angluin et al. in \thref{theorem4}.
 
 \begin{theorem}\thlabel{theorem4}
    \textit{Let $\tau$ be the time at which $x \geq n - \sqrt{n}, y \geq n - \sqrt{n}$, or $v \leq \sqrt{n}$ first holds. Let $v_0$ be the initial number of $x$'s and $y$'s. Then for any fixed $c > 0$ and sufficiently large $n$, if $c_0 \geq \sqrt{n} + c \log_7n$, then} \fda{too many ``then''}
    \begin{align}
        \Pr [ \tau_* \geq 6769n \log n + 6773 c n \log n + 2552n \: \: or \: \: v_{\tau} \leq \sqrt{n}] = n^{-c+o(1)}.  \label{angluinTheorem4} 
    \end{align}
 \end{theorem} 
 
%% ---------------------------------------------------
%% RESULTS FROM GENERALIZATION PAPER
\subsection{Generalization of the 3-state approximate majority protocol}

 \subsubsection{Introduction}
Let $k$ be the amount of potential states in $Q$, excluding the blank state $b$. More formally, $k = |Q \setminus \{b\}|$. The protocol presented by Angluin et al. (\cite{angluinSimplePopulationProtocol2008}) in section \ref{3stateApproximatemajority} is shown to be capable of reaching consensus with a high probability in \inlineMath{O(n \log n)} when $k = 2$. While the convergence rate for this protocol (USD) for $k > 2$ has been studied in \emph{synchronous} systems previously in \cite{becchetti2015}, it was not until very recently that the USD protocol was studied in \emph{asynchronous} systems for $k > 2$. In \cite{AspnesFastConverganceOfKOpinion2023}, Aspnes et al. manages to show that under some mild assumptions it is possible to present a bound on the convergence of the (USD) protocol in a asynchronous system. 

\subsubsection{Notations}

In general most notations in this paper are similar to the ones found in section \ref{notations_angluin_usd}, however due to the fact that \cite{AspnesFastConverganceOfKOpinion2023} explores higher dimensions as a consequence of $k > 2$, some additional notation is required.

A configuration $C_t$ at time $t$ in the execution of the protocol has a corresponding configuration vector $\textbf{x}(t)$. Let the configuration $C_t$ at time $t$ be a represented by a vector $\textbf{x}(t) = (x_1(t), x_2(t), ..., x_k(t), u(t))$. The length of $\textbf{x}(t)$ is $k + 1$. $x_i(t)$ represents the amount of agents in the system with state $i$, where $1 \leq i \leq k$. The amount of agents with the blank state $b$ is represented by $u(t) = n - \sum_{i=1}^k x_i(t)$. When $t = 0$ we assume that $x_1(t) \geq x_2(t) \geq ... \geq x_k(t)$. 

When $t > 0$ we call the index of the state with the largest population $max(t)$. The amount of agents of the largest state at time $t$ we define as $x_{max}(t)$. 

A state $i$ from $Q$ is called \emph{significant} if at time $t$, $x_i(t) > x_max(t) - \alpha \cdot \sqrt{n} \log n$, where $\alpha$ is some fixed constant. If a state $i$ is not significant, it is called \emph{insignificant}. If for a configuration $C$ there exists a state $s$ such that for all other states $s \neq i$ we have $x_s \geq x_i + \beta$, configuration $C$ has a \emph{additive bias} $\beta$. Similarly, if there exists a state $s$ in configuration configuration $C$ where for all other states $s \neq i$, 
$x_s \geq x_i \cdot \alpha$ holds true, the configuration $C$ has a \emph{multiplicative bias} $\alpha$. 

\subsubsection{Results}

In \cite{AspnesFastConverganceOfKOpinion2023}, Aspnes et al. bounds the convergence time of the undecided state dynamics protocol under some assumptions. In order to bound the protocol in terms of $k$, they assume $x_1(0) > n/(2k)$. \thref{theorem5} is a slightly modified version of the main theorem in \cite{AspnesFastConverganceOfKOpinion2023} to match the preliminaries in this paper.

 \begin{theorem}\thlabel{theorem5}
    \textit{Let $c > 0$ be an arbitrary constant and let $\textbf{x}(0)$ be an initial configuration with $k \leq c \cdot \sqrt{n}/\log^2(n) $ opinions with $u(0) \leq (n - x_1(0))/2 and x_1 \geq x_i(0) \ \forall i \in [k]$. Then \cite{AspnesFastConverganceOfKOpinion2023} proves that all agents agree on state 1 within}

    \begin{enumerate}
        \item \textit{$O(n \log n + n^2 / x_1(0)) = O(n \log n + n \cdot k)$ interactions if }$\textbf{x}(0)$ \textit{has a multiplicative bias  of at least $1 + \varepsilon$ for an arbitrary constant $\varepsilon$.}
        \item \textit{$O(n^2 \log n/x_1(0)) = O(k \cdot n \log n)$ interactions if} $\textbf{x}(0)$ \textit{has a additive bias  of at least $\Omega(\sqrt{n} \log n)$.}
    \end{enumerate}

    \textit{Without any bias all agents agree on a significant opinion within $O(n^2 \log n / x_1(0) = O(k \cdot n \log n$ interactions}
 \end{theorem} 



%% END TEXT %%
\clearpage

%% Recent Population protocols
\section{Optimizations of population protocols}
%% TEXT %%
The population protocol in section \ref{Section3} solved the approximate majority problem. This, however, limits the use cases of the protocol as it requires some specific initial bias to converge to the correct solution.  In this section a stable nonuniform population protocol, presented by Doty et al. in \cite{dotyTimeSpaceOptimal2022}, that solves the \emph{exact} majority problem when $k = 2$. 

\subsection{Notations}

A \emph{nonuniform} population protocol is one where the set of transitions used for a specific population size $n$, depends on the value $\lceil \log n \rceil$. Essentially this means that in a nonuniform protocol, for different population sizes $n$, different pairs of $Q$ and $\delta$ are allowed (up to $\lceil \log n \rceil$ combinations). In \cite{dotyTimeSpaceOptimal2022}, all of these combinations combined are referred to as a single protocol. 

Initially each agent has a \emph{bias}: $+1$ for state $x$ and $-1$ for state $y$. Let the \emph{initial gap} be the sum over the entire population $g = \sum_v v.bias$. $g$ is maintained as an invariant. 

Let the constant $L = \lceil \log n \rceil$ be the amount of different values (or biases) an agent in the system can store. For an agent to be able to store $L$ different values, $\log \log n \ + \ O(1)$ bits of memory is needed. Through the agents interactions with each other, they modify their state to any a value in the following set: $\{ 0, \pm \frac{1}{2}, \pm \frac{1}{4}, ...,  \pm \frac{1}{2^L} \}$. The interactions happen through two types of interactions: \emph{cancel reactions} $(+\frac{1}{2^i}, -\frac{1}{2^i}) \mapsto (0, 0)$ and \emph{split reactions} $(\pm \frac{1}{2^i}, 0) \mapsto (\pm \frac{1}{2^{i + 1}}, \pm \frac{1}{2^{i + 1}})$.

The \emph{gap} in the protocol is defined by the sum over the entire population: \\ \mbox{$\sum_v$ sign($v.bias$)}.

Let $c$ be an agent in a sub-population of clock agents. \emph{Drip reactions} for the sub-population are the following: $(c_i, c_i) \mapsto (c_i, c_{i + 1})$. \emph{Epidemic reactions} for the same population are: $(c_i, c_j) \mapsto (c_{max(i, j)}, c_{max(i, j)})$.

\subsection{High-level overview}

The main idea of the protocol is having agents interact with each other through cancel and split reactions. The cancel and split reactions average the bias between the two agents interacting with each other, however only when the resulting average is a power of 2 or 0. The acceptable biases are any value in the following set $\{ 0, \pm \frac{1}{2}, \pm \frac{1}{4}, ...,  \pm \frac{1}{2^L} \}$. $L$ ensures that $\Theta(\log n)$ possible values of \emph{bias} that any agent must be able to store. If this was not the case, and all averages were accepted, then all bias would converge to $\frac{g}{n}$. This would also consume more than the allowed $\log \log n$ bits of memory per agent. The unbiased agents that are required for the split reactions are partially synchronized with a field \emph{hour}. The addition of the field \emph{hour} requires $\log n$ more memory, for $0_0, 0_1, 0_2, ..., 0_L$. The split reaction with the hour field included looks like
\begin{align}
    (\pm \frac{1}{2_i}, 0_h) \mapsto (\pm \frac{1}{2^{i + 1}}, \pm \frac{1}{2^{i + 1}}) \text{      if, } h > i.  \label{splitreaction}
\end{align}
The new split reaction \ref{splitreaction} will also hold until $hour \geq h$ before executing a split reaction that results in $bias = \pm \frac{1}{2^h}$. In \cite{dotyTimeSpaceOptimal2022}, a fast clock using $O(1)$ time per hour is used. The aforementioned \emph{hour} field, used by unbiased agents is synchronized to a different population (still part of $n$) of clock agents, that instead of \emph{hour}, use the field \emph{minute}. In one \emph{hour} there are $k$ consecutive $minutes$. Within this population of clock agents, $minutes$ tick forward using drip reactions and catch up using epidemic reactions. Due to $O(1)$ not being enough time to synchronize all agents, only a large constant of agents will be synchronized to the latest \emph{hour}. Doty et al. proves in \cite{dotyTimeSpaceOptimal2022} that even though not all agents are up to date, the synchronization keeps the $hour$ and $bias$ sufficiently concentrated.

To clean up all agents with the incorrect state, the protocol uses a new population of agents, called \emph{Reserve} agents, that enable more split reactions for agents with large bias values of $|bias| > \frac{1}{2^l}$. After this and after cancel reactions with the majority agents, all agents with the minority state still left must have $|bias| < \frac{1}{2^{l + 2}}$. 

At this point there are still agents with the minority state left that have a small bias. To get rid of these, other agents that have larger bias \emph{consume} them. In \cite{dotyTimeSpaceOptimal2022}, the example of agents with bias $+ \frac{1}{4}$ and $- \frac{1}{256}$ is used. In this example the positive agent can be seen as "holding" the entire bias $+ \frac{1}{4} - \frac{1}{256} = + \frac{63}{256}$. This value is, however, not in the accepted values. Due to this, it cannot longer participate in future averaging interactions. In \cite{dotyTimeSpaceOptimal2022}, however, it is shown that with high probability there are enough majority agents to eliminate all of the remaining minority agents through the previously described \emph{consumption reactions}.

The final part of the protocol checks for positive and negative bias. If one has been removed completely, the system stabilizes to the correct output. If there still are both positive and negative bias, some error has occured.

In the case of a tie, meaning equal size of state $x$ and $y$ in the initial configuration, this protocol detects it with high probability. In an initial configuration where a tie is present, $g = 0$. And with high probability, Doty et al. shows in \cite{dotyTimeSpaceOptimal2022}, that all agents will finish with $bias \in \{ 0, \pm \frac{1}{2^L} \}$. 


\subsection{Results} \label{Section4Results}

The main theorem in \cite{dotyTimeSpaceOptimal2022} is slightly modified to create the following theorem: 

 \begin{theorem}\thlabel{theorem6}
    \textit{
        There is a nonuniform population protocol Nonuniform-Majority, using agents capable of storing $O(\log n)$ different values, that stably computes majority in $O(\log n)$ stabilization time, both in expectation and with high probability.
    }
 \end{theorem} 

The protocol that Doty et al. presents in \cite{dotyTimeSpaceOptimal2022} is capable of quickly solving the exact majority problem in $O(\log n)$ time with high probability using $\log \log n + O(1)$ bits of memory to store $O(\log n)$ different values. The protocol is nonuniform, meaning that all agents must have an estimate of $\lceil \log n \rceil$ embedded in the transition function. Essentially this means that the amount of values $L$ any agent must be capable of storing is a function of $n$. Doty et al. also discusses how the protocol would behave as a uniform protocol, however the protocol would no longer be space optimal due to it needing the memory to store $O(\log n \log \log n)$ different values in each agent. 

In comparison to the undecided state dynamics protocol presented by Angluin et al. in \cite{angluinSimplePopulationProtocol2008} and discussed in section \ref{Section3}, the Nonuniform-Majority protocol solves the exact majority problem, making it much more usable. It is also as fast, reaching consensus with high probability in $O(\log n)$ time, as the USD protocol while solving the exact majority problem. The drawback of the quickness is the use of memory, as the Nonuniform-Majority protocol uses $O(\log \log n)$ bits of memory.  The protocol is also limited to $k = 2$, meaning the decision is only based on two input states. 

% ------------------------------------------------
% GENERALIZAITON
\subsection{Generalization}

In a recent paper by Bankhamer et al. (\cite{bankhamerPopulationProtocolsExact2022}), the protocol discussed in \ref{Section4Results} was extended to support $k > 0$ initial states. While it is known that any always correct protocol requires memory capable of storing $\Omega(k^2)$ values per agent, Bankhamer et al. reduce this drastically by allowing some insignificant failure probability \cite{ongaroSearchUnderstandableConsensus}. 

\subsubsection{Protocols}

In \cite{bankhamerPopulationProtocolsExact2022} presents a protocol for solving the exact majority problem for an arbitrary number $k$ states, where $k > 2$. Three protocols are presented. The first one, called \emph{simple algorithm}, relies on ordering of the states from 1 to $k$ to eliminate non-majority in $k-1$ tournaments. The second protocol removes the reliance on the ordering of states, but sacrifices some time making it slower to reach consensus. The final protocol, called \emph{improved algorithm}, removes insignificant states before staring the tournaments, cutting time to consensus significantly. 

The simple algorithm will be explained briefly. The protocol performs a sequence of \emph{tournaments}, that each take $O(\log n)$ time. These tournaments are synchronized by a \emph{phase clock}. In a tournament, two states are chosen and an exact majority protocol is used to determine the which one of the states populations are larger (i.e. which one has majority). The tournaments begin from state 1 and 2 and work their way up until the last state when the majority will be the majority of the whole system. Described more formally, if $i > 1$ is the number of tournaments performed, the winner of the previous tournament $i - 1$, called \emph{defender}, will be put in a tournament against state $i + 1$, called \emph{challenger}. 

The improved algorithm works in a similar fashion, however, implements a mechanism to get rid of insignificant states before starting the iterations of tournaments. Very briefly, before every the tournament iteration starts, every agent has a counter that counts interactions with the agents with the same state. The first counter to reach a fixed value $t \in O(\log n)$ trigger the beginning of the tournaments. All agents with a counter of under $\frac{t}{2}$ will not participate in the tournament iterations. This removal of insignificant states lowers the required amount of tournaments to $O(\frac{n}{x_{max}})$. This drastically improves the time of the protocol. More detailed descriptions of both algorithm can be found in \cite{bankhamerPopulationProtocolsExact2022}.

\subsubsection{Results}


%% END TEXT %%
% \clearpage

%% Conclusion
\section{Conclusion} 
%% TEXT %%
While consensus algorithms have been researched since the eighties, population protocols are a rather new research domain. There has however been lots of progress in the research of population protocols recently. Angluin et al. introduced the undecided state dynamics protocol (3-state approximate majority protocol) in \cite{angluinSimplePopulationProtocol2008} that solved the approximate majority problem for a system with $k = 2$ states within $O(n \log n)$ interactions with high probability under the constraint that the initial majority $\omega(\sqrt{n} \log n)$. The USD protocol is limited by its initial majority requirement as well as $k = 2$. The latter was solved in \cite{AspnesFastConverganceOfKOpinion2023} by Aspnes et al. when they presented a generalization of the USD protocol allowing it to converge quickly with $k > 2$. The generalized USD protocol using mild assumptions about $k$ solves with high probability the approximate majority consensus problem in $O(k n \log n)$ interactions with an initial additive bias of $\Omega(\sqrt{n} \log n)$. In the case of an multiplicative bias the convergence time for the protocol is improved. 


Conclusion wrapping up the results from section \ref{angluinIntroduction}. Add consideration (Byzantine  etc.). 
Still to be researched, uniform protocols. 


%% REFERENCES
\bibliographystyle{plain}
\bibliography{ref}

\clearpage

%% ADD POSSIBLE APPENDIX HERE
\thesisappendix

\end{document}
