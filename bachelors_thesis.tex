\documentclass[english, 12pt, a4paper, sci, utf8, a-1b, online]{aaltothesis}



\usepackage[parfill]{parskip}
\usepackage{float}
\usepackage{graphicx}
%% Math fonts, symbols, and formatting
\usepackage{amsfonts,amssymb,amsbsy,amsmath, enumitem}
\usepackage{xcolor}
\usepackage{amsthm}
\usepackage{svg}

\newtheorem{theorem}{Theorem}[section]
\usepackage{theoremref} 
\newtheorem{corollary}[theorem]{Corollary}
\newtheorem{lemma}[theorem]{Lemma}
\newtheorem{proposition}[theorem]{Proposition}
\newtheorem{result}[theorem]{Result}
%Definition environments:
\theoremstyle{definition}
\newtheorem{definition}[theorem]{Definition}
%Remark environments:
\theoremstyle{remark}
\newtheorem{remark}[theorem]{Remark}%\declaretheorem{Definition}
%\declaretheorem{Lemma}
%\declaretheorem{Theorem}

%% THESIS INFO
\degreeprogram{Computer Science}

\major{Computer Science}

\code{SCI3028.kand}

\univdegree{BSc}

\thesisauthor{Linus Jern}

\thesistitle{Distributed consensus by population protocols}

\place{Espoo}

\date{\today}

\supervisor{Postdoctoral researcher  \ Francesco d'Amore}

\advisor{Postdoctoral researcher  \ Francesco d'Amore}


%% Aaltologo: syntax:
%% \uselogo{aaltoRed|aaltoBlue|aaltoYellow|aaltoGray|aaltoGrayScale}{?|!|''}
\uselogo{aaltoBlue}{''}


%% THE ENGLISH ABSTRACT:
%% Thesis keywords:
\keywords{Population protocol\spc Consensus Problem\spc Distributed algorithms\spc Majority consensus\spc Byzantine agreement}

%% The abstract text. This text is included in the metadata of the pdf file as well
%% as the abstract page.


\copyrighttext{Copyright \noexpand\copyright\ \number\year\ \ThesisAuthor}
{Copyright \copyright{} \number\year{} \ThesisAuthor}

%------------------------------------------------------------------------------------------------

\newcommand{\linus}[1]{{\color{red} \textbf{Linus}: #1}}

\begin{document}

%% Create the coverpage
\makecoverpage


%% Typeset the copyright text.
\makecopyrightpage

\supervisor{University Lecturer  \ Henrik Wallén}


\pagenumbering{gobble}

%% ENGLISH ABSTRACT
\begin{abstractpage}[english]
    The field of population protocols has seen recent progress in solving the consensus problem. In this literature review some of the significant population protocols solving these issues, as well as recent improvements to these are discussed. The Undecided State Protocol solves the approximate majority problem in $O(n \log n)$ time with high probability provided that the initial majority is at least $\omega(\sqrt{n} \log n)$. It is however limited to $k = 2$, which is two possible opinions. Recent progress improved this protocol to solve the approximate majority problem with $k > 2$ initial opinions. Additionally, the Nonuniform-Majority protocol that solves the exact majority problem for $k = 2$ with high probability is discussed. This protocol is capable of reaching consensus in $O(\log n)$ time, however, the memory usage is not constant using $O(\log n)$ states ($\log \log n + O(1)$ bits of memory). The generalization of this protocol to solve the exact majority problem for $k > 2$ is also discussed.
\end{abstractpage}

%% Force new page so that the Swedish abstract starts from a new page
\newpage

%% SWEDISH ABSTRACT.
\thesistitle{Distribuerad konsensus med hjälp av populations protokoller}
\supervisor{Universitets Lektor  \ Henrik Wallén}
\advisor{Postdoktoral forskare  \ Francesco d'Amore} 
\degreeprogram{Datateknik}
\major{Datateknik}
\keywords{Befolkningsprotokoll\spc Konsensusproblem\spc Distribuerade algoritmer\spc Majoritetsövernskommelse\spc Byzantine övernskommelse}
\begin{abstractpage}[swedish]
    Sammandrag på svenska
\end{abstractpage}

%% TABLE OF CONTENTS
\thesistableofcontents

\cleardoublepage

\pagenumbering{arabic}
%% INTRODUCTION
\section{Introduction}
\thispagestyle{empty}

\subsection{Background}

%% Intro
A distributed system is informally described as a collection of computing entities, also called agents, connected in a shared network which perform computations and exchange messages with each other in order to reach some global task \cite{Coulouris_systems_2005}. Distributed systems are a crucial part of the modern digital infrastructure that the world heavily relies on. They enable large, scalable and low-latency systems for a wide range of applications and audiences. One such system is a distributed database where the responsibility of storing the data is shared across a set of computers. To achieve a shared goal among the agents, they often need to reach a common agreement. An agreement here means that all agents involved have reached a decision on whether to do something or not. Distributed consensus is a fundamental problem in distributed computing that asks the distributed system to achieve an agreement respecting some properties. Consensus is crucial for reaching and maintaining data consistency, ensuring fault tolerance and enabling cooperation between the agents in the system. \cite{Lynch_distributed_2017}

%% Byzantine generals problem
A common way of formalizing the problem of distributed consensus is through the Byzantine Generals Problem, introduced by Leslie Lamport in \cite{lamportByzantineGeneralsProblem}. In this definition there is a number of generals that are trying to coordinate an attack on a city. Some of the generals may prefer to attack and some may prefer to retreat, and they have to reach a shared agreement on whether to attack or not. Additionally, there may be further complications due to treasonous generals spreading sub optimal information. There is also the complication brought by the fact that the only way for the generals to talk to each other is to send a messenger on foot, who might get captured or injured while delivering message. 
%% Byzantine fault
The previously mentioned condition that describes the possibility of a part of the system failing (messenger getting injured or mischievous general in Byzantine terms) is referred to as a Byzantine fault. In short, a Byzantine fault is a fault that causes a different results to arrive to different observers. Or when a sending agent sends out some other message than it should. \cite{driscollRealByzantineGenerals2004}. 
This formalization provides an easily understandable base with which one can visualize the different solutions.

%% Syncrounous vs Asynchronous problem
The way in which agents within a distributed system communicate with each other is another important aspect that should be taken into account when handling the distributed consensus problem. The problem may be modelled in a synchronous or an asynchronous system. A synchronous system is a system where agents use a global clock (or they have perfectly synchronized local clocks), while in an asynchronous system, each agent has its own local clock. The agents in the system perform computational tasks when the clock they are referring to for time tracking ticks. Asynchronous systems introduce additional complexity and cannot solve the distributed consensus problem in general \cite{fischerImpossibilityDistributedConsensus}.

%% Existing solutions
There are multiple established algorithms solving the distributed consensus problem, both in the synchronous and asynchronous systems. Examples of established algorithms are Paxos \cite{lamportFastPaxos2006}, Raft \cite{ongaroSearchUnderstandableConsensus} and Practical Synchronous Byzantine Consensus \cite{renPracticalSynchronousByzantine} in the synchronous setting and pBFT \cite{castroPracticalByzantineFault} and population protocols \cite{aspnesIntroductionPopulationProtocols2009} in the asynchronous setting.

%% Brief paxos
The paxos algorithm mentioned above is one of the first algorithms introduced to solve the consensus problem, even with the presence of failures. It was introduced by Leslie Lamport  in \cite{lamportPartTimeParliment1998}. At a high level, the Paxos algorithm is divided into two phases. In the first phase, a proposer agent sends out it's initial value to the other agents, who respond with an acceptation or an rejection. In the second phase, the proposer agent sends out another proposal, including the responses from the first phase, and repeats this process until the majority if the agents in the system have accepted a common value. 

%% Brief Population protocol
More modern consensus algorithms in the asynchronous setting are population protocols. Population protocols are theoretical models used for modelling collections of moving agents that are capable of interacting and performing computation. The purpose of population protocols are for the collection of agents to converge towards a correct output value. The initial input values for the agents are distributed to the agents in the system after which pairs of agents that are close to each other can exchange information. The agents move around in an unpredictable manner, however subject to some fairness constraints and computation, the collection of agents will converge to the correct output state. \cite{aspnesIntroductionPopulationProtocols2009} 

%% Bried reasoning for focusing on PPs
Population protocols are a modern model in distributed computing that were introduced by Dana Angluin in \cite{angluinComputationNetworksPassively2006}. The application of population protocols vary from sensor networks to the interactions of molecules in theoretical chemistry. \cite{aspnesIntroductionPopulationProtocols2009} The wide application possibilities and the amount of recent breakthroughs made in the field (e.g. \cite{dotyTimeSpaceOptimal2022}, \cite{bankhamerPopulationProtocolsExact2022}) makes it highly interesting and is the reason this thesis will, in addition to giving a overview of the consensus problem, focus on population protocols.

\clearpage

\subsection{Thesis objective}
The goal of this thesis is to clearly define what the problem of distributed consensus is, as well as discuss the existing approaches and algorithms to the problem. Afterwards, we will dig into the modern asynchronous population protocol algorithms that have widely investigated in recent years. 

This thesis is divided into two parts. The first part will discuss the introduction of population protocols, their place in the domain of consensus algorithms, as well as go through the first population protocol presented by Angluin \cite{angluinSimplePopulationProtocol2008}, as well as the generalized version of that same protocol \cite{AspnesFastConverganceOfKOpinion2023}. The second part will cover optimizations of the general population protocols. Lots of research about optimizing different dimensions of population protocols has been done in recent years and one optimization protocol will be discussed in this section. 

There are many literature surveys on specific subdomains of the distributed consensus problem, however not many providing a general overview of the complete topic. As the literature on this topic is large, we will cover some important parts of it with a strong focus on population protocols.


\clearpage

%% SYMBOLS AND ABBREVATIONS
%% SYMBOLS AND ABBREVIATIONS
\section{Preliminaries}


\subsection{Big O notation}

 The Big O is a notation describing the execution time of an algorithm, either through memory or time usage. It is the asymptotic upper bound of a function that describes the algorithm. In computer science the Big O notation is used to analyze the time and space complexity of algorithms. and The Big O notation is a function of \inlineMath{n}, where \inlineMath{n} is the number of items handled in the algorithm. Described informally using the equation \inlineMath{f(n) = O(g(n))}, \inlineMath{f(n)} is smaller than some constant multiplied with \inlineMath{g(n)}.

 \begin{definition}
 We write \inlineMath{f(n) = O(g(n))}, if there exists a constant
 \inlineMath{c > 0} and \inlineMath{k > 0}, such that \inlineMath{0 \leq f(n) \leq cg(n) \: \forall \: n \geq k}
 \end{definition}
\subsection{Distributed system}

A distributed system is a set of networked computers, which coordinate their actions through communicating by messages. A distributed system can be modelled as a single coherent system. We can define this system as a set of nodes, connected in a network, that collectively coordinate and execute tasks.

Let the communication network of the distributed system be a graph \inlineMath{G = (V, E)}, where \inlineMath{V} is the set of vertices (or nodes), meaning the computing entities, or processes, of the system and \inlineMath{E} is the set of edges in the system that make up the communication links between the edges.

Each individual node, or process, in the set of nodes \inlineMath{V} has an internal state \inlineMath{S_p = (x_p, y_p)}, where \inlineMath{x_p} is a one-bit \emph{input register} and \inlineMath{y_p} a one-bit \emph{output register} with values in \inlineMath{\{b, 0, 1\}}. Let the state of the system be  \inlineMath{S = (S_1, S_2, ...., S_N)}. The behaviour of the system can be a set of rule of how the nodes interact with each other, altering the individual internal states of nodes and edges. These rules can be formalized as a set of functions \inlineMath{F}, that map the current state of the system \inlineMath{S} to a new state \inlineMath{S'}.

Define a distributed system as a tuple \inlineMath{(V, E, S, F)}, where \inlineMath{V} is the set of vertices (or nodes), \inlineMath{E} is the set of communication links between nodes, \inlineMath{S} is the current state of the system and \inlineMath{F} is the set of functions that define the behaviour of the system.

\subsection{Consensus problem}

The consensus problem asks to design a protocol that requires all computing entities, called agents, in a system, to agree on a binary value. This system of agent may include faulty agents, that may fail or produce faulty messages. The challenge is to make all non-faulty agents to have a shared understanding of the binary value in question, even with the presence of faulty agents. 

In order to reach consensus, each node in the set \inlineMath{V}, which is the set of nodes in the system, begins by \emph{proposing} the value in it's \emph{input register} \inlineMath{x_p}. Let the value proposed be \inlineMath{v_i}. The nodes then communicate and share their initial proposals. The nodes then decide on a decision value \inlineMath{d_i} and sets it in their respective \emph{output registers} \inlineMath{y_p}. The nodes are now in the \emph{decided state}, from which they can no longer return nor change the value of \inlineMath{y_p}. The requirements of a consensus algorithm are that, for each execution of it, these conditions should hold:

\begin{itemize}[label={}]
  \item \emph{Termination:} All correct nodes eventually sets the value of their output register and reach a decided state.
  \item \emph{Agreement:} All correct nodes share the same value in their output registers: if \inlineMath{V_i} and \inlineMath{V_j} are correct nodes and are in the \emph{decided state}, their corresponding states \inlineMath{S_i} and \inlineMath{S_j} share the same \emph{output register} values \inlineMath{y_i = y_j}.
  \item \emph{Integrity:} If all correct nodes proposed the same value, then any correct node that is in the \emph{decided state} has chosen that value.
\end{itemize}

% TODO

% First define it in a classic setting,
% then define with randomized properties


%% Insert explanation on the graph below
In Figure \ref{fig:ConsensusProblem} we can see an example of the explanation above. Two nodes propose \emph{proceed} and the third node proposes \emph{abort}, but crashes thereafter. The two correct nodes that remain decide on \emph{proceed}.
\begin{figure}[H]
    \centering
    \includesvg[width = 0.7\textwidth]{figures/consensus_problem.svg}
    \caption{TODO caption}
    \label{fig:ConsensusProblem}
\end{figure}


Fault tolerance

\subsection{Asynchronous and synchronous systems}

Explain async and sync envs. Impossibility result in both envs

\subsection{Population protocols}
 A population protocol is a theoretical model used for modelling collections of agents capable of moving, interacting and computation. The goal of the protocol is for the collection of agents to converge towards a correct output value. In the basic population protocol model, an input value is distributed to the the collection of agents. Agents have pairwise interaction in the order set by a scheduler, subject to some fairness guarantee. Each agent in the collection is a type of finite state machine and the protocol for the system describes how the the interaction between two agents change their respective states. No failures occur for the agents in the system. The output values of the agents change over time and eventually, they must converge to the correct output value for the set of input values initially distributed to the agents \cite{aspnesIntroductionPopulationProtocols2009}. 

 A protocol is formally defined by
 \begin{itemize}
     \item \inlineMath{Q}, a finite set of possible states for an agent,
     \item $\Sigma$, a finite input alphabeth,
     \item $\zeta$, an input map $\Sigma \mapsto Q$, where $\zeta(\sigma)$ represents the initial state of an agent and the input to that agent is $\sigma$,
     \item $\omega$, an output map $Q \mapsto Y$, where $Y$ is the output range and $\omega(q)$ represents the output value of an agent in state $q$,
     \item $\delta \subseteq Q^4$, a transition relation that describes interactions between agents.
     
 \end{itemize}

A computation following a protocol defined as above, proceeds as follows. Let the system have \inlineMath{n} \emph{agents}, where \inlineMath{n \geq 2}. And let the computation take place in the system mentioned previously. The input value for each agent in the system is a value from $\Sigma$. The initial state is determined by using $\zeta$ on all agents input values. Let the \emph{configuration} of the system be a vector \inlineMath{C} that contains all the states of the agents. 

A protocol is made up of many executions. An execution alters the \emph{configuration} of the system through pairwise interaction between agents. The agents with states \inlineMath{q_1} and \inlineMath{q_2} can change into the states \inlineMath{q_1'} and \inlineMath{q_2'} if the interaction \inlineMath{(q_1, q_2, q_1', q_2')} is in the transition relation $\delta$. This interaction could also be described using the notation $(q_1, q_2) \mapsto (q_1', q_2')$. Interactions are usually asymmetric, meaning one agent acts as the \emph{initiator} and the other acts as the \emph{responder}. In this example \inlineMath{q_1} would be the initiator and \inlineMath{q_2} would be the responder. If \inlineMath{C} and \inlineMath{C'} are \emph{configurations} in the system, \inlineMath{C \rightarrow C'} means that \inlineMath{C'} can be reached from \inlineMath{C} through a single interaction. The previously mentioned \emph{execution} of the protocol is an inifinite sequence of configurations \inlineMath{C_0, C_1, C_2, ...} where \inlineMath{C_0} is the initial configuration and $C_i \rightarrow C_{i+1} \forall i \geq 0$. 


\subsection{Opinion dynamics}


% \linus{formal definitions of key concepts and notations. e.g. DS, DC problem, population protocols. Check Lynch survey}

%% Early populaiton protocols
\section{Population Protocols}
%% TEXT %%
\subsection{Background}
Explain background to problem

Go through results of Angluin Simple Population Protocol \cite{angluinSimplePopulationProtocol2008}

\subsection{Simple Population Protocol}
explain byzantine generals problem in detail with illustrations

\subsection{Generalization}

Go through the results of the recent study that generalizes the results of \cite{angluinSimplePopulationProtocol2008}. \cite{aspnesFastConverganceOfKOpinion2023}

%% END TEXT %%
\clearpage

%% Recent Population protocols
\section{Optimizations of population protocols}
%% TEXT %%
The population protocol in section \ref{Section3} solved the approximate majority problem. This, however, limits the use cases of the protocol as it requires some specific initial bias to converge to the correct solution.  In this section a stable nonuniform population protocol, presented by Doty et al. in \cite{dotyTimeSpaceOptimal2022}, that solves the \emph{exact} majority problem when $k = 2$. 

\subsection{Notation and definitions}

A \emph{nonuniform} population protocol is one where the set of transitions used for a specific population size $n$, depends on the value $\lceil \log n \rceil$. Essentially this means that in a nonuniform protocol, for different population sizes $n$, different pairs of $Q$ and $\delta$ are allowed (up to $\lceil \log n \rceil$ combinations). In \cite{dotyTimeSpaceOptimal2022}, all of these combinations combined are referred to as a single protocol. 

Initially each agent has a \emph{bias}: $+1$ for opinion $x$ and $-1$ for opinion $y$. Let the \emph{initial gap} be the sum over the entire population $g = \sum_v v.bias$. $g$ is maintained as an invariant. 

Let the constant $L = \lceil \log n \rceil$ be the number of different values (or biases) an agent in the system can store. For an agent to be able to store $L$ different values, $\log \log n \ + \ O(1)$ bits of memory are needed. Through the agents' interactions with each other, they modify their opinion to any value in the following set: $\{ 0, \pm \frac{1}{2}, \pm \frac{1}{4}, ...,  \pm \frac{1}{2^L} \}$. The interactions happen through two types of interactions: \emph{cancel reactions} $(+\frac{1}{2^i}, -\frac{1}{2^i}) \mapsto (0, 0)$ and \emph{split reactions} $(\pm \frac{1}{2^i}, 0) \mapsto (\pm \frac{1}{2^{i + 1}}, \pm \frac{1}{2^{i + 1}})$.

The \emph{gap} in the protocol is defined by the sum over the entire population: \\ \mbox{$\sum_v$ sign($v.bias$)}.

Let $c$ be an agent in a sub-population of clock agents. \emph{Drip reactions} for the sub-population are the following: $(c_i, c_i) \mapsto (c_i, c_{i + 1})$. \emph{Epidemic reactions} for the same population are: $(c_i, c_j) \mapsto (c_{max(i, j)}, c_{max(i, j)})$.

\subsection{High-level overview}

The main idea of the protocol is to have agents interact with each other through cancel and split reactions. The cancel and split reactions average the bias between the two agents interacting with each other, however only when the resulting average is a power of 2 or 0. The acceptable biases are any value in the following set $\{ 0, \pm \frac{1}{2}, \pm \frac{1}{4}, ...,  \pm \frac{1}{2^L} \}$. $L$ ensures that $\Theta(\log n)$ possible values of \emph{bias} that any agent must be able to store. If this was not the case, and all averages were accepted, then all bias would converge to $\frac{g}{n}$. This would also consume more than the allowed $\log \log n$ bits of memory per agent. The unbiased agents that are required for the split reactions are partially synchronized with a field \emph{hour}. The addition of the field \emph{hour} requires $\log n$ more memory, for $0_0, 0_1, 0_2, ..., 0_L$. The split reaction with the hour field included looks like
\begin{align}
    (\pm \frac{1}{2_i}, 0_h) \mapsto (\pm \frac{1}{2^{i + 1}}, \pm \frac{1}{2^{i + 1}}) \text{      if, } h > i.  \label{splitreaction}
\end{align}
The new split reaction \ref{splitreaction} will also hold until $hour \geq h$ before executing a split reaction that results in $bias = \pm \frac{1}{2^h}$. In \cite{dotyTimeSpaceOptimal2022}, a fast clock using $O(1)$ time per hour is used. The aforementioned \emph{hour} field, used by unbiased agents is synchronized to a different population (still part of $n$) of clock agents, that instead of \emph{hour}, use the field \emph{minute}. In one \emph{hour} there are $k$ consecutive $minutes$. Within this population of clock agents, $minutes$ tick forward using drip reactions and catch up using epidemic reactions. Due to $O(1)$ not being enough time to synchronize all agents, only a large constant of agents will be synchronized to the latest \emph{hour}. Doty et al. proves in \cite{dotyTimeSpaceOptimal2022} that even though not all agents are up to date, the synchronization keeps the $hour$ and $bias$ sufficiently concentrated.

To clean up all agents with the incorrect opinion, the protocol uses a new population of agents, called \emph{Reserve} agents, that enable more split reactions for agents with large bias values of $|bias| > \frac{1}{2^l}$. After this and after cancel reactions with the majority agents, all agents with the minority opinion still left must have $|bias| < \frac{1}{2^{l + 2}}$. 

At this point, there are still agents with the minority opinion left that have a small bias. To get rid of these, other agents that have larger bias \emph{consume} them. In \cite{dotyTimeSpaceOptimal2022}, the example of agents with bias $+ \frac{1}{4}$ and $- \frac{1}{256}$ is used. In this example, the positive agent can be seen as "holding" the entire bias $+ \frac{1}{4} - \frac{1}{256} = + \frac{63}{256}$. This value is, however, not in the accepted values. Due to this, it can no longer participate in future averaging interactions. In \cite{dotyTimeSpaceOptimal2022}, however, it is shown that with high probability there are enough majority agents to eliminate all of the remaining minority agents through the previously described \emph{consumption reactions}.

The final part of the protocol checks for positive and negative bias. If one has been removed completely, the system stabilizes to the correct output. If there still are both positive and negative biases, some error has occurred.

In the case of a tie, meaning an equal size of opinion $x$ and $y$ in the initial configuration, this protocol detects it with high probability. In an initial configuration where a tie is present, $g = 0$. And with high probability, Doty et al. shows in \cite{dotyTimeSpaceOptimal2022}, that all agents will finish with $bias \in \{ 0, \pm \frac{1}{2^L} \}$. 


\subsection{Results} \label{Section4Results}

The main theorem in \cite{dotyTimeSpaceOptimal2022} is slightly modified to create the following theorem: 

 \begin{theorem}\thlabel{theorem6}
    \textit{
        There is a non-uniform population protocol Nonuniform-Majority, using agents capable of storing $O(\log n)$ different values, that stably computes the majority in $O(\log n)$ stabilization time, both in expectation and with high probability.
    }
 \end{theorem} 

The protocol that Doty et al. presents in \cite{dotyTimeSpaceOptimal2022} is capable of quickly solving the exact majority problem in $O(\log n)$ time with high probability using $\log \log n + O(1)$ bits of memory to store $O(\log n)$ different values. The protocol is nonuniform, meaning that all agents must have an estimate of $\lceil \log n \rceil$ embedded in the transition function. Essentially this means that the number of values $L$ any agent must be capable of storing is a function of $n$. Doty et al. also discuss how the protocol would behave as a uniform protocol, however, the protocol would no longer be space optimal due to it needing the memory to store $O(\log n \log \log n)$ different values in each agent. 

In comparison to the undecided state dynamics protocol presented by Angluin et al. in \cite{angluinSimplePopulationProtocol2008} and discussed in section \ref{Section3}, the Nonuniform-Majority protocol solves the exact majority problem, making it much more usable. It is also faster, reaching consensus with high probability in $O(\log n)$ time, as the USD protocol while solving the exact majority problem. The drawback of the quickness is the use of memory, as the Nonuniform-Majority protocol uses $O(\log \log n)$ bits of memory.  The protocol is also limited to $k = 2$, meaning the decision is only based on two input opinions. 
Additionally, due to the non-uniformity, the protocol is dependent on the input size $n$, and the agents need to know this. This essentially means all agents in the system need to know how many agents there are in the system before starting the execution, which is not wanted in many cases. 


% ------------------------------------------------
% GENERALIZAITON
\subsection{Generalization}

In a recent paper by Bankhamer et al. (\cite{bankhamerPopulationProtocolsExact2022}), the protocol discussed in \ref{Section4Results} was extended to support $k > 0$ initial opinions.
While it is known that any always correct protocol requires memory capable of storing $\Omega(k^2)$ states per agent, Bankhamer et al. reduce this drastically by allowing some insignificant failure probability \cite{ongaroSearchUnderstandableConsensus}. 

\subsubsection{Protocols} \label{441}

In \cite{bankhamerPopulationProtocolsExact2022} presents a protocol for solving the exact majority problem for an arbitrary number $k$ opinions, where $k > 2$. Three protocols are presented. The first one called \emph{simple algorithm}, relies on the ordering of the states from 1 to $k$ to eliminate non-majority in $k-1$ tournaments. The second protocol removes the reliance on the ordering of states but sacrifices some time making it slower to reach consensus. The final protocol, called \emph{improved algorithm}, removes insignificant states before starting the tournaments, cutting time to consensus significantly. 

The simple algorithm performs a sequence of \emph{tournaments}, that each takes $O(\log n)$ time. These tournaments are synchronized by a \emph{phase clock}. In a tournament, two opinions are chosen and an exact majority protocol is used to determine which one of the opinions populations is larger (i.e. which one has majority). The tournaments begin from opinions 1 and 2 and work their way up until the last opinion when the majority will be the majority of the whole system. Described more formally, if $i > 1$ is the number of tournaments performed, the winner of the previous tournament $i - 1$, called \emph{defender}, will be put in a tournament against opinion $i + 1$, called \emph{challenger}. 

The improved algorithm works in a similar fashion, however, implements a mechanism to get rid of insignificant opinions before starting the iterations of tournaments. Very briefly, before every tournament iteration starts, every agent has a counter that counts interactions with the agents with the same opinion. The first counter to reach a fixed value $t \in O(\log n)$ triggers the beginning of the tournaments. All agents with a counter of under $\frac{t}{2}$ will not participate in the tournament iterations. This removal of insignificant opinions lowers the required amount of tournaments to $O(\frac{n}{x_{max}})$. This drastically improves the time of the protocol. More detailed descriptions of both algorithms can be found in \cite{bankhamerPopulationProtocolsExact2022}.

\subsubsection{Results}

The main result Bankhamer et al. presents in \cite{bankhamerPopulationProtocolsExact2022} is the improved algorithm briefly described in section \ref{441}. The following theorem is a slightly modified version (to match preliminaries and notations of this paper) of the main result of \cite{bankhamerPopulationProtocolsExact2022}

 \begin{theorem}\thlabel{theorem7}
    \textit{
        Assume we have a population of size $n$ with $k$ initial opinions where $x_{max} > n^{\frac{1}{2 + \varepsilon}}$ for some small constant $\frac{1}{2} > \varepsilon > 0$. The \emph{improved algorithm} converges with high probability to the majority opinion in $O(\frac{n}{x_{max} \cdot \log n + \log^2 n})$ time while requiring agents to be capable of storing $O(k \cdot \log \log n + \log n)$ different values.}
 \end{theorem} 

The \emph{improved algorithm} in \cite{bankhamerPopulationProtocolsExact2022} expands the Nonuniform-Majority protocol from \thref{theorem6} (from \cite{dotyTimeSpaceOptimal2022}) to create a protocol for $k > 2$ number of initial opinions. This protocol solves the exact majority problem with high probability quickly while only requiring the agents to be able to store $O(k + \log n)$ values. 


%% Conclusion
\section{Conclusion} 
%% TEXT %%
\linus{TODO: Wrap up results from previous two sections}

Conclusion wrapping up the results from section \ref{angluinIntroduction}. Add consideration (Byzantine  etc.). 
Still to be researched, uniform protocols. 


%% REFERENCES
\bibliographystyle{plain}
\bibliography{ref}

\clearpage

%% ADD POSSIBLE APPENDIX HERE
\thesisappendix

\end{document}
