\section{Population Protocols}
%% TEXT %%
\subsection{Background}
\linus{TODO: Explain background to consensus problem and its relation to population protocols}

The consensus problem is fundamental challenge in the field of distributed computing. The problem has been studied extensively in literature and there are multiple popular algorithms 


\subsection{3-state approximate majority protocol} \label{3stateApproximatemajority}
%% ---------------------------------------------------
%% INTRODUCTION
\subsubsection{Introduction} \label{angluinIntroduction}

One of the first population protocol for approximate majority consensus was presented for a simple 3-state system by Angluin et al. in \cite{angluinSimplePopulationProtocol2008}. The protocol is shown to converge to a consensus in \inlineMath{O(n \log n)} interactions with high probability. It is also shown that the output value is correct, meaning it matches the initial majority, with high probability if the initial net majority is $\omega(\sqrt{n} \log n)$. \fda{Maybe define with high probability?}

The protocol is a simple 3-state protocol, meaning \inlineMath{Q = \{x, y, b\}}. The purpose of the protocol is to make all agents decide on the initial majority state, either \inlineMath{x} and \inlineMath{y}. The extra state \inlineMath{b} is a an \emph{blank} state. The idea of the protocol is that when two agents with different states meet in an interaction, the \emph{responder} abandons its state and enters the blank state \inlineMath{b}. If an agent with the blank state \inlineMath{b} interacts with another agent, it adopts the other agents state, assuming the other agent does not also have the blank state \inlineMath{b}. Because the collisions between agents are chosen by an adversary, the interactions between agents with the opposite state are equally balanced. However, because an agent with the blank state is more likely to interact with an agent with the initial majority state, the initial majority will increase until all agents have the same state. Once all agents have converged to the same state, the protocol is finished and the system has reached consensus. This protocol is also known as the \emph{Undecided State Dynamics} or \emph{USD} protocol.

Furthermore, Angluin shows that with high probability that the inclusion of \inlineMath{o(\sqrt{n})} Byzantine agents cannot notably delay the protocol converging to a state where the majority of non-Byzantine agents have the correct state. Byzantine agents are agents capable of appearing as any state, regardless of their previous interaction and initial state. The inclusion of these Byzantine agents can, however, keep a small part of of the non-Byzantine agents confused. Additionally, after exponential time on average, the Byzantine agents can eventually push the system to a stable incorrect state, where all of the non-Byzantine agents are blank. In the case that \inlineMath{z} Byzantine agents are included amongst the population \inlineMath{n} of normal agents, any execution select a pair at random from the combined population of \inlineMath{n} and \inlineMath{z}.

\subsubsection{Notations} \label{notations_angluin_usd}

The agents in the system can have the states in \inlineMath{Q = \{x, y, b\}}, where \inlineMath{b} is the \emph{blank} state. We can map out all possible interactions between agents of different states in \inlineMath{Q} in the following table

\begin{figure}[H]
    \centering
    \begin{tabular}{|c | c | c | c|} 
     \hline
      & \inlineMath{x} & \inlineMath{b} & \inlineMath{y} \\ [0.5ex] 
     \hline
     \inlineMath{x} & \inlineMath{(x, x)} & \inlineMath{(x, x)} & \inlineMath{(x, b)} \\ 
     \hline
     \inlineMath{b} & \inlineMath{(b, x)} & \inlineMath{(b, b)} & \inlineMath{(b, y)} \\
     \hline
     \inlineMath{y} & \inlineMath{(y, b)} & \inlineMath{(y, y)} & \inlineMath{(y, y)} \\
     \hline
    \end{tabular}
    \caption{Potential interactions for \inlineMath{Q}}
    \label{fig:QInteractions}
\end{figure}

All interactions modify the \emph{responders} state. This makes the protocol \emph{one-way}. Notable is also that not all interactions changes the states. For example \inlineMath{xx} does not change any state. The interactions that do change states are \inlineMath{xy, yx, yb, xb}.The system can be in three stable configurations \inlineMath{C}: all \inlineMath{b}'s, all \inlineMath{x}'s or all \inlineMath{y}'s. The first one of the previously named configurations cannot be reached from any configuration containing anything else than \inlineMath{b}'s. The two latter configurations are also stable and from every configuration made up of not only \inlineMath{b}'s, one of them can be reached.

Let \inlineMath{x_t, y_t} and $b_t$ be the number of \inlineMath{x, y} and \inlineMath{b} after \inlineMath{t} interactions. Let the \emph{convergence time} $\tau_*$ of the protocol be the first time \inlineMath{t} at which \inlineMath{x_t = n} or \inlineMath{y_t = n}.

To reduce size of explanations, we also define:
\begin{description}
    \centering
    \item[] \inlineMath{u = x - y}
    \item[] \inlineMath{v = x + y = n - b}
    \item[] \inlineMath{g = 1/n(n - 1)}
\end{description}

This definition of $u$ and $v$ exposes the symmetry between $x$ and $y$. The variable $g$ gives us the opportunity to calculate the probability of an interaction happening. For example $gvb$ gives the probability of a non-blank agent interacting with a blank agent. We define a configuration space with four regions: a central space where the $x, y$ and $b$ are quite evenly balanced and three corner regions where the amount of agents with status $x, y$ or $b$, depending on the corner, is almost $n$. The configuration space is visualized in Figure \ref{fig:configurationSpace}.

\begin{figure}[H]
    \centering
    \includesvg[width = 0.65\textwidth]{figures/configuration_space.svg}
    \caption{Configuration space}
    \label{fig:configurationSpace}
\end{figure}


To define interactions and sums of interactions, indicator variables are used. For example, let $I^{vb}_t$ be any interaction between a $xy$ or $yx$ pair at time $t$. All indicator variables $I_t$ have a corresponding sum variable $S_t = \sum^t_{i=1}I_i$ that describes the total amount of times this indicator event has occurred up until time $t$. The following indicator variables will be used:

\begin{table}[H]
    \centering
    \begin{tabular}{|c c c|}
     \hline
     Indicator & Sum & Event \\ 
     \hline
     \inlineMath{I^{vb}} & \inlineMath{S^{vb}} & $xb$ or $yb$ interaction \\
     \hline
     \inlineMath{I^{xy}} & \inlineMath{S^{xy}} & $xy$ or $yx$ interaction \\
     \hline
     \inlineMath{I^{b}} & \inlineMath{S^{b}} & Interaction in $b$ corner with $b \geq (7/8)n$ \\
     \hline
     \inlineMath{I^{x}} & \inlineMath{S^{x}} & Interaction in $x$ corner with $x \geq (7/8)n$  \\
     \hline
     \inlineMath{I^{y}} & \inlineMath{S^{y}} & Interaction in $y$ corner with $y \geq (7/8)n$  \\
     \hline
     \inlineMath{I^{c}} & \inlineMath{S^{c}} & Central interaction: $I^b = I^x = I^y = 0$ \\
     \hline
     \inlineMath{I^{z}} & \inlineMath{S^{z}} & Interaction with a Byzantine agent \\
     \hline
    \end{tabular}
    \caption{Interaction and sum notations}
    \label{fig:QInteractions}
\end{table}

%% ---------------------------------------------------
%% RESULTS FROM SIMPLE PROPROT
\subsubsection{Results}

The main results that Angluin et al. presents in \cite{angluinSimplePopulationProtocol2008} were briefly described in section \ref{angluinIntroduction}. The protocol itself is very simple, agents interacting according to the rules set in figure \ref{fig:QInteractions} will converge towards a consensus. By the proofs presented in \cite{angluinSimplePopulationProtocol2008}, a high-probability bound for the total interactions done before reaching convergence can be presented with the following theorem.

 \begin{theorem}\thlabel{theorem1}
    \textit{Let $\tau_*$ be the time at which \inlineMath{x = n} or \inlineMath{y = n} first holds. Then for any fixed \inlineMath{c > 0} and sufficiently large $n$,}
    \begin{align}
        \centerline{$\Pr[\tau_* \geq \log n + 6773cn \log + 2552n] \leq 5n^{-c}$.}  \label{angluinTheorem1} 
    \end{align}
 \end{theorem}

\ref{angluinTheorem1} in \thref{theorem1} gives us is a high probability upper bound for $\tau_*$. The constants in this theorem are large and it is shown in \cite{angluinSimplePopulationProtocol2008} that simulations produce much smaller constants, however, theorem \ref{theorem1} gives us a theoretical upper bound for the protocol in a system with only non-Byzantine agents. This convergence is quite quick, but does however require a initial margin between the population size of $x$ and $y$ in the initial state of the system. The next main result Angluin et al. shows in \cite{angluinSimplePopulationProtocol2008} is the lower bound for the difference between initial population sizes that the protocol tolerates, still producing a high probability convergence. This lower bound is presented in \thref{theorem2}


 \begin{theorem}\thlabel{theorem2}
    \textit{With high probability, the 3-state approximate majority protocol converges to the initial majority value if the difference between the initial majority and initial minority populations is } $\omega(\sqrt{n} \log n)$.
 \end{theorem} 

 Angluin also explores the correctness of the protocol in the case where the protocol has an epidemic-triggered start in \cite{angluinSimplePopulationProtocol2008}. The difference here compared to the systems previously discussed is that agents have an additional \emph{active/inactive} property. The initial state contains some subset of active and some subset of inactive agents. The inactive agents gets recruited into the computation of the protocol by active agents. It is shown in \cite{angluinSimplePopulationProtocol2008} that to guarantee a convergence to the correct value, a larger initial majority is needed, meaning a larger lower bound. The third main Angluin et al. presents in \cite{angluinSimplePopulationProtocol2008} is \thref{theorem3}.

 \begin{theorem}\thlabel{theorem3}
    \textit{Let $\epsilon > 0$. If the difference between the initial majority and initial minority populations is $\Omega(n^{3/4+\epsilon})$ and there is exactly one active agent, then with high probability, the epidemic-triggered approximate majority protocol converges to the initial value.}
 \end{theorem} 

 The inclusion of Byzantine agents in the system is also studied in \cite{angluinSimplePopulationProtocol2008}. A new group of agents $z$ is introduced into the system, where $z = o(\sqrt{n})$. Angluin shows that, even with the presence of $z$ Byzantine agents, the protocol will converge in $O(n \log n)$ interactions. This, however, requires some modifications to both the initial majority requirements as well as the definition of convergence. Due to the fact that any Byzantine agent at any time can shift the state to something that would not be possible in the presence of only non-Byzantine agents. A Byzantine agent could shift to a $y$ agent while there are no non-Byzantine $y$ agents left. This forces the convergence acceptance criteria to be slackened a bit, allowing some non-Byzantine agents to be in the incorrect state. Additionally, there exists a possibility where the inclusion of Byzantine agents leads the system to a stable state with only $b$. Angluin shows that even with these additional complexities, the probability of reaching the $b$ corner (as visualized in Figure \ref{fig:configurationSpace}) is small and that reaching the $x$ and $y$ corners still does not require that many interactions. The convergence time with the inclusion of Byzantine agents is presented by Angluin et al. in \thref{theorem4}.
 
 \begin{theorem}\thlabel{theorem4}
    \textit{Let $\tau$ be the time at which $x \geq n - \sqrt{n}, y \geq n - \sqrt{n}$, or $v \leq \sqrt{n}$ first holds. Let $v_0$ be the initial number of $x$'s and $y$'s. Then for any fixed $c > 0$ and sufficiently large $n$, if $c_0 \geq \sqrt{n} + c \log_7n$, then} \fda{too many ``then''}
    \begin{align}
        \Pr [ \tau_* \geq 6769n \log n + 6773 c n \log n + 2552n \: \: or \: \: v_{\tau} \leq \sqrt{n}] = n^{-c+o(1)}.  \label{angluinTheorem4} 
    \end{align}
 \end{theorem} 
 
%% ---------------------------------------------------
%% RESULTS FROM GENERALIZATION PAPER
\subsection{Generalization of the 3-state approximate majority protocol}

 \subsubsection{Introduction}
Let $k$ be the amount of potential states in $Q$, excluding the blank state $b$. More formally, $k = |Q \setminus \{b\}|$. The protocol presented by Angluin et al. (\cite{angluinSimplePopulationProtocol2008}) in section \ref{3stateApproximatemajority} is shown to be capable of reaching consensus with a high probability in \inlineMath{O(n \log n)} when $k = 2$. While the convergence rate for this protocol (USD) for $k > 2$ has been studied in \emph{synchronous} systems previously in \cite{becchetti2015}, it was not until very recently that the USD protocol was studied in \emph{asynchronous} systems for $k > 2$. In \cite{AspnesFastConverganceOfKOpinion2023}, Aspnes et al. manages to show that under some mild assumptions it is possible to present a bound on the convergence of the (USD) protocol in a asynchronous system. 

\subsubsection{Notations}

In general most notations in this paper are similar to the ones found in section \ref{notations_angluin_usd}, however due to the fact that \cite{AspnesFastConverganceOfKOpinion2023} explores higher dimensions as a consequence of $k > 2$, some additional notation is required.

A configuration at a specific time $t$ in the execution of the protocol $C_t$, will now also be refereed to as $\textbf{x}(t)$. Let the configuration $C_t$ at time $t$ be a represented by a vector $\textbf{x}(t) = (x_1(t), x_2(t), ..., x_k(t), u(t))$. The length of $\textbf{x}(t)$ is $k + 1$. $x_i(t)$ represents the amount of agents in the system with state $i$, where $1 \leq i \leq k$. The amount of agents with the blank state $b$ is represented by $u(t) = n - \sum_{i=1}^k x_i(t)$. 

\subsubsection{Results}
Present results of \cite{AspnesFastConverganceOfKOpinion2023}

%% END TEXT %%
\clearpage