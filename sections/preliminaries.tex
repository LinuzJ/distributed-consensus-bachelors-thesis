%% SYMBOLS AND ABBREVIATIONS
\mysection{Preliminaries}


\subsection{Big O notation}

 The Big O is a notation describing the execution time of an algorithm, either through memory or time usage. It is the asymptotic upper bound of a function that describes the algorithm. In computer science the Big O notation is used to analyze the time and space complexity of algorithms. and The Big O notation is a function of \inlineMath{n}, where \inlineMath{n} is the number of items handled in the algorithm. Described informally using the equation \inlineMath{f(n) = O(g(n))}, \inlineMath{f(n)} is smaller than some constant multiplied with \inlineMath{g(n)}.

 \begin{definition}
 We write \inlineMath{f(n) = O(g(n))}, if there exists a constant
 \inlineMath{c > 0} and \inlineMath{k > 0}, such that \inlineMath{0 \leq f(n) \leq cg(n) \: \forall \: n \geq k}
 \end{definition}
\subsection{Distributed system}

A distributed system is a set of networked computers, which coordinate their actions through communicating by messages. A distributed system can be modelled as a single coherent system. We can define this system as a set of nodes, connected in a network, that collectively coordinate and execute tasks.

Let this system be a graph \inlineMath{G = (E, N)} \francesco{the graph is the communication network underlying the distributed system. Usually,m one writes $G = (V,E)$ where $V$ is the set of vertices (or nodes), and $E$ the set of edges.}, where \inlineMath{N} represents the nodes, meaning the computing entities of the system and \inlineMath{E} represents the edges in the system that make up the communication links between the edges.

\francesco{Before talking about the state of the system, we must define it.}
Let \inlineMath{S} be the state of the system. This state may represent either the state of individual nodes, or the communication links between these edges. The behaviour of the system can be a a set of rule of how the nodes interact with each other, altering the local state of nodes and edges. These rules can be formalized as a set of functions \inlineMath{F}, that map the current state of the system \inlineMath{S} to a new state \inlineMath{S'}.

Thus, a distributed system can be defined as the tuple \inlineMath{(N, E, S, F)}, where \inlineMath{N} is the set of nodes, \inlineMath{E} is the set of communication links between nodes, \inlineMath{S} is the current state of the system and \inlineMath{F} is the set of functions that define the behaviour of the system.

\subsection{Consensus problem}


The consensus problem is the problem of trying to get \francesco{I would say that the problem asks to design a \emph{protocol} that requires...} all computing entities, called agents, in a system, to agree on a binary value. This system of agent may include faulty agents, that may fail or produce faulty messages. The challenge is to make all non-faulty agents to have a shared understanding of the binary value in question, even with the presence of faulty agents. 

Majority consensus

Fault tolerance

\subsection{Asynchronous and synchronous systems}

Explain async and sync envs. Impossibility result in both envs

\subsection{Population protocols}



\subsection{Opinion dynamics}


% \linus{formal definitions of key concepts and notations. e.g. DS, DC problem, population protocols. Check Lynch survey}