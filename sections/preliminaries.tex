%% SYMBOLS AND ABBREVIATIONS
\mysection{Preliminaries}


\subsection{Big O notation}
 The Big O notation is a measure of the execution time of an algorithm, either through memory or time usage. It is the asymptotic upper bound of a function that describes the algorithm. In computer science the Big O notation is used to analyze the time and space complexity of algorithms. and The Big O notation is a function of \inlineMath{n}, where \inlineMath{n} is the number of items handled in the algorithm. Described informally using the equation \inlineMath{f(n) = O(g(n))}, \inlineMath{f(n)} is smaller than some constant multiplied with \inlineMath{g(n)}.

 \begin{Definition}
 \inlineMath{f(n) = O(g(n))} describes an equation where there exists positive constants \inlineMath{c} and \inlineMath{k}, such that \inlineMath{0 \leq f(n) \leq c*g(n) \: \forall \: n \geq k}
 \end{Definition}
 
\subsection{Distributed systems}


\subsection{Consensus problem}

\subsection{Asynchronous and synchronous systems}

\subsection{Population protocols}

\subsection{Opinion dynamics}

\begin{itemize}
    \item notations
    \item definitions
\end{itemize}

\linus{formal definitions of key concepts and notations. e.g. DS, DC problem, population protocols. Check Lynch survey}