\section{Introduction}
\thispagestyle{empty}

\subsection{Background}
%% TEXT %%
\linus{change all nodes to agents}
%% Intro
A distributed system is informally described as a collection of computing entities, also called agents, connected in a shared network which perform computations and exchange messages with each other in order to reach some global task \cite{Coulouris_systems_2005}. Distributed systems are a crucial part of the modern digital infrastructure that the world heavily relies on. They enable large, scalable and low-latency systems for a wide range of applications and audiences. \linus{Add example here}. To achieve a shared goal among the agents, they often need to reach a common agreement. \linus{expand here what agreement means} Distributed consensus is a fundamental problem in distributed computing that asks the distributed system to achieve an agreement respecting some properties. Consensus is crucial for reaching and maintaining data consistency, ensuring fault tolerance and enabling cooperation between the nodes in the system. \linus{citations and examples for the last sentence}

%% Byzantine generals problem
A common way of formalizing the problem of distributed consensus is through the Byzantine Generals Problem, introduced by Leslie Lamport in \cite{lamportByzantineGeneralsProblem}. In this definition there is a number of generals that are trying to coordinate an attack on a city. Some of the generals may prefer to attack and some may prefer to retreat, and they have to reach a shared agreement on whether to attack or not. Additionally, there may be further complications due to treasonous generals spreading suboptimal information. There is also the complication brought by the fact that the only way for the generals to talk to each other is to send a messenger on foot, who might get captured or injured while delivering message. 
%% Byzantine fault
The previously mentioned condition that describes the possibility of a part of the system failing (messenger getting injured or mischievous general in Byzantine terms) is referred to as a Byzantine fault. In short, a Byzantine fault is any fault that makes different agents in the system see different results \cite{driscollRealByzantineGenerals2004}. \linus{improve}
This formalization provides an easily understandable base with which one can visualize the different solutions.

%% Syncrounous vs Asynchronous problem
The way in which agents within a distributed system communicate with each other is another important aspect that should be taken into account when handling the distributed consensus problem. The problem may be modelled in a synchronous or an asynchronous system. A synchronous system is a system where agents use a global clock (or they have perfectly synchronized local clocks), while in an asynchronous system, each agent has its own local clock. \linus{add that agents perform on clockrate} Asynchronous systems introduce additional complexity and cannot solve the distributed consensus problem in general \cite{fischerImpossibilityDistributedConsensus}.

%% Existing solutions
There are multiple established algorithms solving the distributed consensus problem, both in the synchronous and asynchronous systems. Examples of established algorithms are Paxos \cite{lamportFastPaxos2006}, Raft \cite{ongaroSearchUnderstandableConsensus} and Practical Synchronous Byzantine Consensus \cite{renPracticalSynchronousByzantine} in the synchronous setting and pBFT \cite{castroPracticalByzantineFault} and population protocols \cite{aspnesIntroductionPopulationProtocols2009} in the asynchronous setting.

\linus{add paragraph, introducing a more in depth description for the algos mention above. Informally descibe population protocols. What is paxos? in descriptive way -> running time and fault tolerance. Add reason for focusing on pp}

\clearpage

\linus{paper -> theis}
\subsection{Thesis objective}
The goal of this paper is to clearly define what the problem of distributed consensus is, as well as discuss the existing approaches and algorithms to the problem. Afterwards, we will dig into the modern asynchronous population protocol algorithms that have widely investigated in recent years. 

This paper is divided into three parts. The first two parts will define what the distributed consensus problem is and what type of algorithms there exists to this problem. The third part will describe what the modern population protocol algorithms are, as well as give an overview of the recent research done in this field.

There are many literature surveys on specific subdomains of the distributed consensus problem, however not many providing a general overview of the complete topic. As the literature on this topic is large, we will cover some important parts of it with a strong focus on population protocols.
\clearpage