\section{Introduction}
\thispagestyle{empty}

\subsection{Background}
%% TEXT %%

% OLD INTRO
% In this bachelor's thesis the problem of distributed consensus and the different approaches to solving it are discussed. Distributed consensus in a distributed system means reaching a common agreement among the agents involved in said system. There are both classical and more modern approaches to solving this problems in both a synchronous state, as well as in a asynchronous state. 

% The existing material and research in this field is vast, both through books and academic research. Some specific names stand out as for example Nancy Lynch and her book Distributed Algorithms \cite{distributed_algorithms}, as well as Leslie Lamport work on formalization of the Byzantine problem \cite{byzantine_generals}. There is also good material on modern approaches, such as population protocols done by for example James Aspnes \cite{simple_populaiton_protocol}.

% • broad introduction to the topic, talk about general research directions that
% have been investigated (be general here and cite works) -> 1 page
% • thesis objective: talk a bit more about the questions and the problems that
% you will address in the thesis + implication in the field -> 0,5 page
% • start explaining a bit more in details these topics (definitions, results, difficulties)
% -> 1 page
% • end of introduction: some consideration that you might have, connections with
% other works, open questions (either existing or that you can come up with),
% etc -> 0,5 page

Distributed systems are a crucial part of the modern digital infrastructure that the world heavily relies on. They enable large, scalable and low-latency systems for a wide range of applications and audiences. Distributed systems can be large and complex with a large amount of agents working together. To enable achieve a shared goal amongst the agents, a consensus is required on desired input and output. Distributed consensus is a fundamental concept in distributed systems that describes to process of reaching consensus among multiple agents in a network. It is crucial for reaching and maintaining data consistency, ensuring fault tolerance and enabling cooperation between the agents in the system. 

\vspace{1em}

A common way of formalizing the problem of distributed consensus is through what's called the Byzantine Generals Problem. This formalization describes the problem through a case where a number of generals are trying to coordinate an attack on a city. Some of the generals may prefer to attack and some may prefer to retreat, and they have to reach a shared agreement on whether to attack or not. Additionally, there may be further complications due to treasonous generals spreading sub optimal information. There is also the complication brought by the fact that the only way for the generals to talk to each other is to send a messenger on foot, who might get captured or injured, to delivery message.  \cite{Lamport_Byzantine_1982}

asdadasd
\clearpage

\subsection{Thesis objective}


\subsection{}
\clearpage