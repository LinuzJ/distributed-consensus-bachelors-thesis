\section{Introduction}
\thispagestyle{empty}

\subsection{Background}
%% TEXT %%

%% Intro
Distributed systems are a crucial part of the modern digital infrastructure that the world heavily relies on. They enable large, scalable and low-latency systems for a wide range of applications and audiences. Distributed systems can be vast and complex with a large amount of nodes working together. To achieve a shared goal amongst the nodes, a consensus is required on desired input and output. Distributed consensus is a fundamental concept in distributed systems that describes to process and the protocol for reaching consensus among multiple nodes in a network. Consensus here means having a shared understanding amongst the nodes of what to do. Consensus is crucial for reaching and maintaining data consistency, ensuring fault tolerance and enabling cooperation between the nodes in the system. 

%% Byzantine generals problem
A common way of formalizing the problem of distributed consensus is through the Byzantine Generals Problem, introduced by Leslie Lamport. This formalization describes the problem through a case where a number of generals are trying to coordinate an attack on a city. Some of the generals may prefer to attack and some may prefer to retreat, and they have to reach a shared agreement on whether to attack or not. Additionally, there may be further complications due to treasonous generals spreading sub optimal information. There is also the complication brought by the fact that the only way for the generals to talk to each other is to send a messenger on foot, who might get captured or injured, to delivery message \cite{lamportByzantineGeneralsProblem}. 
%% Byzantine fault
The previously mentioned condition that describes the possibility of a part of the system failing (messenger getting injured or mischievous general in Byzantine terms) is refereed to as a Byzantine fault \cite{lamportByzantineGeneralsProblem}. In short, a Byzantine fault is any fault which makes different observers see different results \cite{driscollRealByzantineGenerals2004}.
This formalization provides an easily understandable base with which you can visualize the different solutions.

%% Syncrounous vs Asynchronous problem
The way in which processes within a distributed system communicate with each other is another important aspect that should be taken into when viewing the distributed consensus problem. The distributed consensus problem may be modelled as a synchronous or asynchronous system. A synchronous system is a system where each process runs using a global clock or perfectly synchronized clocks, while there is no global clock nor consistent clock rate in a asynchronous system.


\clearpage

\subsection{Thesis objective}
The goal of this paper is to identify and clearly define what the problem of distributed consensus is,
as well as discuss the existing approaches and solutions to the problem. After this this paper will look into the modern asynchronous population protocol algorithms that have been the subject of many papers in the recent years. 
This paper will give a overview of the current knowledge of distributed consensus algorithms, as well as a brief overlook on where and how these might be applied in practice.

The paper is a literature study that is divided into three parts. The first two parts will describe what the distributed consensus problem is and what type of solutions there exists to this problem. After this, the third part will describe why the modern population protocol solutions have been developed, as well as give an overview of the recent research done in this field. In this paper the following research questions will be used as support:

\begin{itemize}
    \item Clearly define what the distributed consensus problem is
    \item To provide an overview of the current state of the art in distributed consensus algorithms and their applications in various domains.
    \item Distinguish the differences between the modern and classical approaches
    \item Deep dive into modern population protocol algorithms
\end{itemize}

Due to the fact that the algorithms and topic are quite complex, the algorithms in this paper will be discussed and compared on quite a high level.

\subsection{}

\clearpage