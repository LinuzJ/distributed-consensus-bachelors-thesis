\section{Introduction}
\thispagestyle{empty}
%% Intro

\subsection{Background}
A distributed system can be described as a collection of computing entities, also called agents, connected in a shared network that perform computations and exchange messages with each other in order to reach some global task \cite{Coulouris_systems_2005}. Distributed systems are a crucial part of the modern digital infrastructure that the world heavily relies on. They enable large, scalable, and low-latency systems for a wide range of applications and audiences. One such system is a distributed database where the responsibility of storing the data is shared across a set of computers. To achieve a shared goal among the agents, they often need to reach a common agreement. An agreement here means that all agents involved have reached a decision on whether to do something or not. Distributed consensus is a fundamental problem in distributed computing that asks the distributed system to achieve an agreement respecting some properties. Consensus is crucial for reaching and maintaining data consistency, ensuring fault tolerance, and enabling cooperation between the agents in the system. \cite{Lynch_distributed_2017}

%% Byzantine generals problem
A common way of formalizing the problem of distributed consensus is through the Byzantine Generals Problem, introduced by Leslie Lamport in \cite{lamportByzantineGeneralsProblem}. In this definition, there is several generals trying to coordinate an attack on a city. Some of the generals may prefer to attack and some may prefer to retreat, and they have to reach a shared agreement on whether to attack or not. Additionally, there may be further complications like treasonous generals spreading sub-optimal information and the fact that the only way for the generals to talk to each other is to send a messenger on foot, who might get captured or injured while delivering the message. 
%% Byzantine fault
The previously mentioned condition that describes the possibility of a part of the system failing (messenger getting injured or mischievous general in Byzantine terms) is referred to as a Byzantine fault. In short, a Byzantine fault is a fault that causes a different result to arrive to different observers. Or when a sending agent sends out some other message than it should. \cite{driscollRealByzantineGenerals2004}. 
This formalization provides an easily understandable base with which one can visualize the different solutions.

%% Synchronous vs Asynchronous problem
The way in which agents communicate within a distributed system is another important aspect that should be taken into account when handling the distributed consensus problem. The problem may be modeled in a synchronous or asynchronous system. A synchronous system is a system where agents use a global clock (or they have perfectly synchronized local clocks), while in an asynchronous system, each agent has its own local clock. The agents in the system perform computational tasks when the clock they are referring to ticks one time unit forward. Asynchronous systems introduce additional complexity and cannot solve the distributed consensus problem in general \cite{fischerImpossibilityDistributedConsensus}.

%% Existing solutions
To solve the consensus problem, there are multiple established algorithms, both in synchronous and asynchronous systems. Examples of established algorithms are Paxos \cite{lamportFastPaxos2006}, Raft \cite{ongaroSearchUnderstandableConsensus} and Practical Synchronous Byzantine Consensus \cite{renPracticalSynchronousByzantine} in the synchronous setting and pBFT \cite{castroPracticalByzantineFault} and population protocols \cite{aspnesIntroductionPopulationProtocols2009} in the asynchronous setting.

%% Brief paxos
The Paxos algorithm mentioned above is one of the first algorithms introduced to solve the consensus problem, even with the presence of failures. It was introduced by Leslie Lamport  in \cite{lamportPartTimeParliment1998}. At a high level, the Paxos algorithm is divided into two phases. In the first phase, a proposer agent sends out its initial value to the other agents, who respond with an acceptation or an rejection. In the second phase, the proposer agent sends out another proposal, including the responses from the first phase, and repeats this process until the majority of the agents in the system have accepted a common value. 

%% Brief Population protocol
Population protocols are more modern consensus algorithms in the asynchronous setting. Population protocols are theoretical models used for modeling collections of moving agents that are capable of interacting and performing computation. Their purpose is for the collection of agents to converge toward a correct output value. The initial input values for the agents are distributed to the agents in the system after which adjacent pairs of agents can exchange information. The agents move around in an unpredictable manner, however subject to some fairness constraints and computation, the collection of agents will converge to the correct output state. \cite{aspnesIntroductionPopulationProtocols2009} 

%% Bried reasoning for focusing on PPs
Population protocols are a modern model in distributed computing that was introduced by Dana Angluin in \cite{angluinComputationNetworksPassively2006}. The application of population protocols varies from sensor networks to the interactions of molecules in theoretical chemistry. \cite{aspnesIntroductionPopulationProtocols2009} The wide application possibilities and the number of recent breakthroughs made in the field (e.g. \cite{dotyTimeSpaceOptimal2022}, \cite{bankhamerPopulationProtocolsExact2022}) make population protocols highly interesting and is the reason this thesis will, in addition to giving an overview of the consensus problem, focus on population protocols.


\subsection{Thesis objective}
The goal of this thesis is to clearly define what the problem of distributed consensus is, as well as discuss the existing approaches and algorithms to the problem, focusing on population protocols.

This thesis is divided into two parts. The first part discusses the introduction of population protocols, and their place in the domain of consensus algorithms, as well as goes through the first population protocol presented by Angluin \cite{angluinSimplePopulationProtocol2008}, as well as the generalized version of that same protocol \cite{AspnesFastConverganceOfKOpinion2023}. The second part will cover a space and time-optimized population protocol (\cite{dotyTimeSpaceOptimal2022}) as well as a generalized version (\cite{bankhamerPopulationProtocolsExact2022}) of that protocol. 
