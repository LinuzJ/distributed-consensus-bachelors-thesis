\section{Introduction}
\thispagestyle{empty}

\subsection{Background}

%% Intro
A distributed system is informally described as a collection of computing entities, also called agents, connected in a shared network which perform computations and exchange messages with each other in order to reach some global task \cite{Coulouris_systems_2005}. Distributed systems are a crucial part of the modern digital infrastructure that the world heavily relies on. They enable large, scalable and low-latency systems for a wide range of applications and audiences. One such system is a distributed database where the responsibility of storing the data is shared across a set of computers. To achieve a shared goal among the agents, they often need to reach a common agreement. An agreement here means that all agents involved have reached a decision on whether to do something or not. Distributed consensus is a fundamental problem in distributed computing that asks the distributed system to achieve an agreement respecting some properties. Consensus is crucial for reaching and maintaining data consistency, ensuring fault tolerance and enabling cooperation between the agents in the system. \cite{Lynch_distributed_2017}

%% Byzantine generals problem
A common way of formalizing the problem of distributed consensus is through the Byzantine Generals Problem, introduced by Leslie Lamport in \cite{lamportByzantineGeneralsProblem}. In this definition there is a number of generals that are trying to coordinate an attack on a city. Some of the generals may prefer to attack and some may prefer to retreat, and they have to reach a shared agreement on whether to attack or not. Additionally, there may be further complications due to treasonous generals spreading sub optimal information. There is also the complication brought by the fact that the only way for the generals to talk to each other is to send a messenger on foot, who might get captured or injured while delivering message. 
%% Byzantine fault
The previously mentioned condition that describes the possibility of a part of the system failing (messenger getting injured or mischievous general in Byzantine terms) is referred to as a Byzantine fault. In short, a Byzantine fault is a fault that causes a different results to arrive to different observers. Or when a sending agent sends out some other message than it should. \cite{driscollRealByzantineGenerals2004}. 
This formalization provides an easily understandable base with which one can visualize the different solutions.

%% Syncrounous vs Asynchronous problem
The way in which agents within a distributed system communicate with each other is another important aspect that should be taken into account when handling the distributed consensus problem. The problem may be modelled in a synchronous or an asynchronous system. A synchronous system is a system where agents use a global clock (or they have perfectly synchronized local clocks), while in an asynchronous system, each agent has its own local clock. The agents in the system perform computational tasks when the clock they are referring to for time tracking ticks. Asynchronous systems introduce additional complexity and cannot solve the distributed consensus problem in general \cite{fischerImpossibilityDistributedConsensus}.

%% Existing solutions
There are multiple established algorithms solving the distributed consensus problem, both in the synchronous and asynchronous systems. Examples of established algorithms are Paxos \cite{lamportFastPaxos2006}, Raft \cite{ongaroSearchUnderstandableConsensus} and Practical Synchronous Byzantine Consensus \cite{renPracticalSynchronousByzantine} in the synchronous setting and pBFT \cite{castroPracticalByzantineFault} and population protocols \cite{aspnesIntroductionPopulationProtocols2009} in the asynchronous setting.

%% Brief paxos
The paxos algorithm mentioned above is one of the first algorithms introduced to solve the consensus problem, even with the presence of failures. It was introduced by Leslie Lamport  in \cite{lamportPartTimeParliment1998}. At a high level, the Paxos algorithm is divided into two phases. In the first phase, a proposer agent sends out it's initial value to the other agents, who respond with an acceptation or an rejection. In the second phase, the proposer agent sends out another proposal, including the responses from the first phase, and repeats this process until the majority if the agents in the system have accepted a common value. 

%% Brief Population protocol
More modern consensus algorithms in the asynchronous setting are population protocols. Population protocols are theoretical models used for modelling collections of moving agents that are capable of interacting and performing computation. The purpose of population protocols are for the collection of agents to converge towards a correct output value. The initial input values for the agents are distributed to the agents in the system after which pairs of agents that are close to each other can exchange information. The agents move around in an unpredictable manner, however subject to some fairness constraints and computation, the collection of agents will converge to the correct output state. \cite{aspnesIntroductionPopulationProtocols2009} 

%% Bried reasoning for focusing on PPs
Population protocols are a modern model in distributed computing that were introduced by Dana Angluin in \cite{angluinComputationNetworksPassively2006}. The application of population protocols vary from sensor networks to the interactions of molecules in theoretical chemistry. \cite{aspnesIntroductionPopulationProtocols2009} The wide application possibilities and the amount of recent breakthroughs made in the field (e.g. \cite{dotyTimeSpaceOptimal2022}, \cite{bankhamerPopulationProtocolsExact2022}) makes it highly interesting and is the reason this thesis will, in addition to giving a overview of the consensus problem, focus on population protocols.

\clearpage

\subsection{Thesis objective}
The goal of this thesis is to clearly define what the problem of distributed consensus is, as well as discuss the existing approaches and algorithms to the problem. Afterwards, we will dig into the modern asynchronous population protocol algorithms that have widely investigated in recent years. 

This thesis is divided into two parts. The first part will discuss the introduction of population protocols, their place in the domain of consensus algorithms, as well as go through the first population protocol presented by Angluin \cite{angluinSimplePopulationProtocol2008}, as well as the generalized version of that same protocol \cite{AspnesFastConverganceOfKOpinion2023}. The second part will cover optimizations of the general population protocols. Lots of research about optimizing different dimensions of population protocols has been done in recent years and one optimization protocol will be discussed in this section. 

There are many literature surveys on specific subdomains of the distributed consensus problem, however not many providing a general overview of the complete topic. As the literature on this topic is large, we will cover some important parts of it with a strong focus on population protocols.
