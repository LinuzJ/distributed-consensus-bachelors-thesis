\section{Conclusion} 
%% TEXT %%
While consensus algorithms have been researched since the eighties, population protocols are a rather new research domain. There has however been lots of progress in the research of population protocols recently. Angluin et al. introduced the undecided state protocol (3-state approximate majority protocol) in \cite{angluinSimplePopulationProtocol2008} that solved the approximate majority problem for a system with $k = 2$ states within $O(n \log n)$ interactions with high probability under the constraint that the initial majority $\omega(\sqrt{n} \log n)$. The USD protocol is limited by its initial majority requirement as well as $k = 2$. The latter was solved in \cite{AspnesFastConverganceOfKOpinion2023} by Aspnes et al. when they presented a generalization of the USD protocol allowing it to converge quickly with $k > 2$. The generalized USD protocol using mild assumptions about $k$ solves with high probability the approximate majority consensus problem in $O(k n \log n)$ interactions with an initial additive bias of $\Omega(\sqrt{n} \log n)$. In the case of an multiplicative bias the convergence time for the protocol is improved. The undecided state dynamics population protocol solves the approximate majority problem, however for solving the exact majority problem other protocol are needed. The Nonuniform-Majority protocol presented by Doty et al. in \cite{dotyTimeSpaceOptimal2022} solves the exact majority problem for $k = 2$ initial states in optimal space and time complexity, that is in $O(\log n)$ time and requiring each agent to be able to store $O(\log n)$ values ($\log \log n + O(1)$) bits of memory. Quite quickly after Doty et al. realeased \cite{dotyTimeSpaceOptimal2022}, Bankhammer et al. \cite{bankhamerPopulationProtocolsExact2022} built on that protocol to create a generalization of the Nonuniform-Majority protocol that can solve the exact majority problem for $k > 2$ initial states. This protocol is capable of solving the exact majority problem within $O(\frac{n}{x_{max} \cdot \log n + \log^2 n})$ interactions while requiring agents to be capable of storing $O(k \cdot \log \log n + \log n)$ values. 


\linus{Conclusion wrapping up the results from section \ref{angluinIntroduction}. Add consideration (Byzantine  etc.). 
Still to be researched, uniform protocols. }
